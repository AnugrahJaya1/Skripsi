%versi 2 (8-10-2016)
\chapter{Landasan Teori}
\label{chap:teori}
\section{Sistem Rekomendasi}
\label{sec:sistem rekomendasi}

Sistem rekomendasi adalah alat dan teknik perangkat lunak yang menyediakan saran untuk item yang akan digunakan oleh pengguna. Saran terkait dengan berbagai proses pengambilan keputusan, seperti barang apa yang akan dibeli, musik apa yang akan didengarkan, atau berita online apa yang akan dibaca. % RS hanbook

Sistem rekomendasi biasanya berfokus pada item tertentu seperti buku,musik,dll. Sistem rekomendasi ditujukan untuk individu atau personal yang kurang memiliki pengalaman pribadi. Contoh Sistem rekomendasi buku adalah \textit{website} Amazon.com. Item yang ditawarkan sebagai daftar item peringkat. Sistem rekomendasi mencoba memprediksi produk dengan cara mengumpulkan referensi dari pengguna lainnya.\\ %RS hanbook

Pengembangan sistem rekomendasi dimulai dari pengamatan yang sederhana berupa rekomendasi yang diberikan oleh orang lain dalam membuat keputusan rutin sehari-hari bisa berupa buku, musik, film, rekrutmen karyawan, dll. %RS hanbook

Sistem rekomendasi menghasilkan rekomendasi menggunakan berbagai jenis pengetahuan dan data tentang pengguna, item yang tersedia, dan transaksi sebelumnya, contohnya berupa e-commerce yang mengatasi masalah kelebihan informasi yang terjadi akibat transaksi pengguna sebelumnya. \\ %RS hanbook

\subsection{Fungsi Sistem Rekomendasi}
Fungsi utama sistem rekomendasi adalah menemukan item yang relevan dengan kebutuhan pengguna.Selain untuk menemukan item yang relevan terdapat juga beberapa fungsing sistem rekomendasi, yaitu  : %RS hanbook
	\begin{enumerate}
		\item Meningkatkan jumlah penjualan barang\\
			Salah satu fungsi penting untuk sistem rekomendasi yang komersil. Peningkatan jumlah penjualan item ini disebabkan karena penjualan item dilakukan tepat sasaran kepada pembeli yang memang membutuhkan dan menginginkan item tersebut. Merekomendasikan item yang sesuai dengan kebutuhan atau minat pengguna.
			
		\item Menjual barang-barang yang lebih beragam\\
			Memberikan rekomendasi item yang mungkin sulit ditemukan oleh pengguna jika tanpa menggunakan sistem rekomendasi.
			
		\item Meningkatkan kepuasan pengguna\\
			Sistem rekomendasi yang dirancang dengan baik memberikan rekomendasi yang sesuai dengan kebutuhan pengguna sehingga pengguna akan merasa senang menggunakan sistem tersebut.
		
		\item Meningkatkan kesetiaan pengguna\\
			Pengguna akan tetap menggunakan sebuah \textit{website} jika sistem rekomendasi yang hasilkan rekomendasi yang sesuai dengan kebutuhan pengguna. 
			
		\item Lebih mengerti apa yang diinginkan pengguna\\
			Sistem dapat memebrikan hasil rekomendasi item yang sesuai dengan kebutuhan pengguna.
	\end{enumerate} \leavevmode

\subsection{Sumber Data dan Pengetahuan}
\label{sec:sumber data dan pengetahuan}
Sistem rekomendasi adalah sistem pemrosesan informasi yang secara aktif mengumpulkan berbagai jenis data untuk membangun rekomendasinya. Data utama berupa data item yang disarankan dan pengguna yang akan menerima rekomendasi. Data yang digunakan sistem rekomendasi mengacu pada tige jenis objek, yaitu : %RS hanbook
	\begin{enumerate}
	\item Item\\
		Item adalah objek yang direkomendasikan, item bisa ditandai oleh kompleksitasnya dan nilai atau kegunaannya. Bisa bernilai positif jika sesuai atau negatif jika tidak sesuai.
	
	\item Pengguna\\
		Pengguna adalah objek yang menggunakan sistem, memiliki tujuan dan karakteristik beragam. Untuk mempersonalisasi rekomendasi, sistem rekomendasi mengeksploitasi berbagai informasi pengguna. Pengguna juga dapat dijelaskan oleh data pola perilaku (pola penelusuran web, atau pola pencarian perjalanan)
	
	\item Transaksi\\
		Interaksi yang direkam antara pengguna sistem rekomendasi. Transaksi adalah data seperti log yang menyimpan informasi penting yang dihasilkan selama interaksi manusia-komputer dan berguna untuk algoritma pembuatan rekomendasi yang digunakan sistem. Bentuk dari peringkat yang populer di sistem rekomendasi :
		
		\begin{itemize}
			\item Peringkat numerik 1 - 5
		
			\item Peringkat ordinal (sangat setuju, setuju, netral, tidak setuju, dan sangat tidak setuju)
		
			\item Peringkat biner, buruk (0) dan baik (1)
		
			\item Peringkat unary menunjukkan bahwa pengguna telah mengamati atau membeli barang atau menilai barang secara positif
		\end{itemize}
		
	\end{enumerate}
	
\subsection{Teknik Rekomendasi}
\label{sec:teknik rekomendasi}
Berikut adalah teknik-teknik yang dapat digunakan pada sistem rekomendasi : \\%RS hanbook
\begin{enumerate}
	\item \textit{Content-based}\\
		Sistem merekomendasikan item yang mirip berdasarkan item yang disukai pengguna di masa lalu. Kesamaan dihitung berdasarkan fitur(atribut) yang terkait dengan item. misal , review positif film komedi, maka akan direkomendasikan film di genre yang sama. 

	\item \textit{Collaborative Filtering} \\
		Implementasi paling sederhana, merekomendasikan item yang disukai pengguna lain dengan selera serupa di masa lalu. CF populer dan banyak digunakan pada RS. Nearest neighbors meningkatkan popularitas karena sederhana, efisien, dan kemampuan mereka untuk menghasilkan rekomendasi yang akurat dan dipersonalisasi.
	
	\item \textit{Demographic} \\
		Rekomendasi berdasarkan profil demografis pengguna. Asumsinya bahwa rekomendasi yang berbeda harus dihasilkan untuk demografis yang berbeda. Misalnya diarahkan ke web dengan bahasa atau negara pengguna. 

	\item \textit{Knowledge-based} \\
		Merekomendasikan item berdasarkan pengetahuan domain spesifik tentang fitur (atribut) item tertentu yang memenuhi kebutuhan atau referensi pengguna. 

	\item \textit{Community-based} \\
		Merekomendasikan item berdasarkan teman-teman pengguna. Bukti menunjukan bahwa orang cenderung lebih mengandalkan rekomendasi dari teman-teman dari pada rekomendasi dari orang yang belum dikenal. 

	\item \textit{Hybrid recomender systems} \\
		Kombinasi dari hal diatas. Menggunakan teknik A dan B mencoba untuk menggunakan keunggulan A dan memperbaiki kelemahan B. misal CF memiliki kelemahan terhadap item yang tidak memiliki peringkat (ga ada histori)  bisa digabungkan dengan metode content-based
\end{enumerate}

% jelasin konsep dasar Collabrative Filtering
\subsection{\textit{Collaborative Filtering}}
\label{sec:collaborative filtering}
	Dalam pengembangan sistem rekomendasi dapat menggunkan teknik \textit{Collaborative Filtering}. \textit{Collaborative Filtering} menghasilkan rekomendasi item yang spesifik untuk pengguna berdasarkan peringkat tanpa memerlukan informasi tambahan mengenai item ataupun pengguna. Gagasan utamanya adalah peringkat pengguna u untuk item i cenderung mirip dengan pengguna v, jika u dan v memberikan peringkat item lain dengan nilai yang sama. \\%RS hanbook
	Tantangan dalam membangun sistem rekomendasi menggunakan teknik \textit{Collaborative Filtering} adalah sedikitnya jumlah data pengguna sebelumnya yang sudah memberikan peringkat kepada suatu item. Dalam \textit{Collaborative Filltering} terdapat salah satu algoritma yaitu \textit{Neighborhood-based Collaborative Filtering} atau yang dikenal dengan \textit{Memory-base Collaborative Filtering}.
	
\subsubsection{\textit{Neighborhood-based Collaborative Filtering}}
textit{Neighborhood-based Collaborative Filtering} atau yang dikenal dengan \textit{Memory-base Collaborative Filtering} adalah algoritma pertama yang dikembangan untuk teknik \textit{Collaborative Filtering}. Algoritma ini peringkat \textit{user-item} disimpan dalam sistem secara langsung digunakan untuk memprediksi peringkat item baru, dapat dilakukan dengan \textit{user-based model}. %RS hanbook

\subsubsection{\textit{•User-based Neighborhood Model}}
\textit{User-based} bekerja dengan mengidentifikasi pengguna yang akan diberikan rekomendasi dengan pengguna yang memiliki kesamaan. Aktivitas pengguna yang memiliki kesamaan ini akan menjadi dasar dalam memberikan rekomendasi kepada pengguna. Aktivitas bisa berupa memberikan peringkat kepada item. Berikut adalah tahapan yang perlu dilakukan pada \textit{User-based} : 

\begin{enumerate}
	\item Mencari nilai rata-rata peringkat yang sudah diberikan oleh pengguna sebelumnya.
	
	\item Menghitung kesamaan atau similaritas pengguna menggunakan \textit{Pearson Correlation Coefficient} 2.1 :
	
	\begin{equation}
		sim(i,j) = Pearson(i,j) = \frac{\Sigma _{k\epsilon} I_{i} \cap I_{j} (r_{i,k}-\mu_{i}) \cdot (r_{j,k}-\mu_{j})}{\sqrt{\Sigma _{k\epsilon} I_{i} \cap I_{j} (r_{i,k}-\mu_{i})^2} \cdot \sqrt{\Sigma _{k\epsilon} I_{i} \cap I_{j} (r_{j,k}-\mu_{j})^2 }}
	\end{equation}\leavevmode \\
	Keterangan : 
	\begin{itemize}
		\item sim(i,j) = Kesamaan atau similaritas antara pengguna i dan pengguna j
		
		\item $\Sigma _{k\epsilon} I_{i} \cap I_{j}$ = Himpunan item pengguna i dan pengguna j yang saling beririsan
		
		\item $r_{i,k}$ = Nilai yang diberikan pengguna i terhadap item k
		
		\item $r_{j,k}$ = Nilai yang diberikan pengguna j terhadap item k
		
		\item $\mu_{i}$ = Rata-rata nilai yang diberikan pengguna i
		
		\item $\mu_{j}$ = Rata-rata nilai yang diberikan pengguna j
	\end{itemize}\leavevmode
	
	\item Mengurutkan nilai kesamaan atau similaritas dari yang terbesar ke terkecil. Memiliki rentan nilai -1, 0, dan +1 untuk pengguna yang akan diberikan rekomendasi. Jika hasil perhitungan mendekati -1, berarti pengguna tersebut kurang memiliki kesamaan dengan pengguna yang akan diberikan rekomendasi. Jika hasil perhitungan mendekati 1, berarti pengguna tersebut memiliki kesamaan yang cukup baik dengan pengguna yang akan diberikan rekomendasi. Jika hasil perhitungan mendekati +1, berarti pengguna tersebut memiliki kesamaan yang tinggi dengan pengguna yang akan diberikan rekomendasi.
	%kayanya bagian ini ga usah, soalnya nnti ambil K tetangga terdekat, liat IPK + jurusan, filteri IPK > 2, trs count jurusannya, rekomendasiin
	\item Menghitung nilai prediksi dengan rumus umum prediksi 2.2 :
	
	\begin{equation}
		r_{i,k} = \mu_{i} + \frac{\Sigma _{j \epsilon} P_{i} Sim(i,j)\cdot (r_{j,k} - \mu_{j})}{\Sigma _{j \epsilon} P_{i(k)} |Sim(i,j)|}
	\end{equation}\leavevmode \\
	
	Keterangan :
	\begin{itemize}
		\item $r_{i,k}$ = Nilai prediksi pengguna i untuk item k
		
		\item Sim(i,j)= kesamaan atau similiaritas pengguna i dan pengguna j
		
		\item $r_{j,k}$ = Penilaian pengguna j terhdap item k
		
		\item $\mu_{j}$ = Nilai rata-rata pengguna j
		
		\item $\mu_{i}$ = Nilai rata-rata pengguna i
	\end{itemize}\leavevmode
	 
\end{enumerate}\leavevmode



\subsection{Aplikasi dan Evalusi}
\label{sec:aplikasi dan evaluasi}
\subsubsection{Aplikasi}
Faktor pertama yang harus dipertimbangkan adalah domain aplikasi, memiliki efek yang besar pada pendekatan algoritmik yang diambil. Kelas domain paling umum : %RS hanbook
\begin{enumerate}
	\item Entertainment : rekomendasi film dan musik
	\item Content : personalisasi berita, dokukumen, dan web page
	\item E-commerce : rekomendasi produk untuk di beli
	\item Services : rekomendasi servis travel, hotel, dan rumah
\end{enumerate}

\subsubsection{Evalusi}
Evaluasi diperlukan untuk sistem rekomendasi. Evaluasi \textit{offline} dilakukan dengan cara menjalankan beberapa algoritma pada data yang sama dan membandingkan kinerjanya. Evaluasi \textit{online} dilakukan saat sistem sudah diluncurkan, melibatkan pengguna nyata. \textit{Focused user study}, jika
evaluasi online tidak layak atau terlalu beresiko, meminta beberapa pengguna untuk mencoba
sistem.

\section{Program Studi}
\label{sec:program studi} 
%Diisi dengan penjelasan program studi dan syarat-syaratnya
Perguruan tinggi adalah satuan pendidikan yang menyelenggarakan pendidikan tinggi yang dapat berbentuk akademi, politeknik, sekolah tinggi, institut. atau universitas. Pendidikan tinggi adalah kelanjutan pendidikan menengah yang
diselenggarakan untuk menyiapkan peserta didik menjadi anggota masyarakat yang memiliki kemampuan akademik dan/atau profesional yang dapat menerapkan, mengembangkan dan/atau menciptakan ilmu pengetahuan. teknologi dan/atau kesenian. \\ %SK MENDIKNAS 232

Program studi adalah kesatuan rencana belajar sebagai pedoman penyelenggaraan pendidikan akademik dan/atau profesional yang diselenggarakan atas dasar suatu kurikulum serta ditujukan agar mahasiswa dapat menguasai pengetahuan, keterampilan, dan sikap sesuai dengan sasaran kurikulum. Kurikulum pendidikan tinggi adalah seperangkat rencana dan pengaturan mengenai isi maupun bahan kajian dan pelajaran serta cara penyampaian dan penilaiannya yang digunakan sebagai pedoman penyelenggaraan kegiatan belajar - mengajar di perguruan tinggi. \\ %SK MENDIKNAS 232



\section{Template Skripsi FTIS UNPAR}
\label{sec:template}
 
Akan dipaparkan bagaimana menggunakan template ini, termasuk petunjuk singkat membuat referensi, gambar dan tabel.
Juga hal-hal lain yang belum terpikir sampai saat ini. 
 
\dtext{15-16}

\subsection{Tabel}  
Berikut adalah contoh pembuatan tabel. 
Penempatan tabel dan gambar secara umum diatur secara otomatis oleh \LaTeX{}, perhatikan contoh di file bab2.tex untuk melihat bagaimana cara memaksa tabel ditempatkan sesuai keinginan kita.

Perhatikan bawa berbeda dengan penempatan judul gambar gambar, keterangan tabel harus diletakkan di atas tabel!!
Lihat Tabel~\ref{tab:contoh1} berikut ini:

\begin{table}[H] %atau h saja untuk "kira kira di sini"
	\centering 
	\caption{Tabel contoh}
	\label{tab:contoh1}
	\begin{tabular}{cccc}
		\toprule
		& $v_{start}$ & $\mathcal{S}_{1}$ & $v_{end}$\\

		\midrule
		$\tau_{1}$ & 1 & 12& 20\\
		$\tau_{2}$ & 1 &  & 20\\
		$\tau_{3}$ & 1 & 9 & 20\\
		$\tau_{4}$ & 1 &  & 20\\

		\bottomrule
		
	\end{tabular} 
\end{table}
Tabel~\ref{tab:cthwarna1} dan Tabel~\ref{tab:cthwarna2} berikut ini adalah tabel dengan sel yang berwarna dan ada dua tabel yang bersebelahan. 
\begin{table}[H]
	\begin{minipage}[c]{0.49\linewidth}
		\centering
		\caption{Tabel bewarna(1)}
		\label{tab:cthwarna1}
		\begin{tabular}{ccccc}
			\toprule
			 & $v_{start}$ & $\mathcal{S}_{2}$ & $\mathcal{S}_{1}$ & $v_{end}$\\
			
			\midrule
			$\tau_{1}$ & 1 & 5 \cellcolor{green}& 12& 20\\
			$\tau_{2}$ & 1 & 8 \cellcolor{green}& & 20\\
			$\tau_{3}$ & 1 & 2/8/17 \cellcolor{green}& 9 & 20\\
			$\tau_{4}$ & 1 & \cellcolor{red}& & 20\\
			
			\bottomrule

		\end{tabular}
	\end{minipage}
	\begin{minipage}[c]{0.49\linewidth}
		
		\centering 
		\caption{Tabel bewarna(2)}
		\label{tab:cthwarna2}
		\begin{tabular}{ccccc}
			\toprule
			 & $v_{start}$ & $\mathcal{S}_{1}$ & $\mathcal{S}_{2}$ & $v_{end}$\\
			
			\midrule
			$\tau_{1}$ & 1 & 12& 5 \cellcolor{red} &20\\
			$\tau_{2}$ & 1 &  &  8 \cellcolor{green} &20\\
			$\tau_{3}$ & 1 & 9 & 2/8/17 \cellcolor{green} &20\\
			$\tau_{4}$ & 1 &   & \cellcolor{red} &20\\
			
			\bottomrule
		
		\end{tabular}
	\end{minipage}
\end{table}

 
\subsection{Kutipan}
\label{subs:kutipan} 
Berikut contoh kutipan dari berbagai sumber, untuk keterangan lebih lengkap, silahkan membaca file referensi.bib yang disediakan juga di template ini.
Contoh kutipan:
\begin{itemize}
	\item Buku:~\cite{berg:08:compgeom} 
	\item Bab dalam buku:~\cite{kreveld:04:GIS}
	\item Artikel dari Jurnal:~\cite{buchin:13:median}
	\item Artikel dari prosiding seminar/konferensi:~\cite{kreveld:11:median}
	\item Skripsi/Thesis/Disertasi:~\cite{lionov:02:animasi}~\cite{wiratma:10:following}~\cite{wiratma:22:later}
	\item Technical/Scientific Report:~\cite{kreveld:07:watertight}
	\item RFC (Request For Comments):~\cite{RFC1654}
	\item Technical Documentation/Technical Manual:~\cite{Z.500}~\cite{unicode:16:stdv9}~\cite{google:16:and7}
	\item Paten:~\cite{webb:12:comm}
	\item Tidak dipublikasikan:~\cite{wiratma:09:median}~\cite{lionov:11:cpoly}
	\item Laman web:~\cite{erickson:03:cgmodel}  
	\item Lain-lain:~\cite{agung:12:tango}
\end{itemize}    
  
\subsection{Gambar}

Pada hampir semua editor, penempatan gambar di dalam dokumen \LaTeX{} tidak dapat dilakukan melalui proses {\it drag and drop}.
Perhatikan contoh pada file bab2.tex untuk melihat bagaimana cara menempatkan gambar.
Beberapa hal yang harus diperhatikan pada saat menempatkan gambar:
\begin{itemize}
	\item Setiap gambar {\bf harus} diacu di dalam teks (gunakan {\it field} {\sc label})
	\item {\it Field} {\sc caption} digunakan untuk teks pengantar pada gambar. Terdapat dua bagian yaitu yang ada di antara tanda $[$ dan $]$ dan yang ada di antara tanda $\{$ dan $\}$. Yang pertama akan muncul di Daftar Gambar, sedangkan yang kedua akan muncul di teks pengantar gambar. Untuk skripsi ini, samakan isi keduanya.
	\item Jenis file yang dapat digunakan sebagai gambar cukup banyak, tetapi yang paling populer adalah tipe {\sc png} (lihat Gambar~\ref{fig:ularpng}), tipe {\sc jpg} (Gambar~\ref{fig:ularjpg}) dan tipe {\sc pdf} (Gambar~\ref{fig:ularpdf})
	\item Besarnya gambar dapat diatur dengan {\it field} {\sc scale}.
	\item Penempatan gambar diatur menggunakan {\it placement specifier} (di antara tanda  $[$ dan $]$ setelah deklarasi gambar.
	Yang umum digunakan adalah {\bf H} untuk menempatkan gambar {\bf sesuai} penempatannya di file .tex atau  {\bf h} yang berarti "kira-kira" di sini. \\
	Jika tidak menggunakan {\it placement specifier}, \LaTeX{} akan menempatkan gambar secara otomatis untuk menghindari bagian kosong pada dokumen anda.
	Walaupun cara ini sangat mudah, hindarkan terjadinya penempatan dua gambar secara berurutan. 	
	\begin{itemize}
		\item Gambar~\ref{fig:ularpng} ditempatkan di bagian atas halaman, walaupun penempatannya dilakukan setelah penulisan 3 paragraf setelah penjelasan ini.
		\item Gambar~\ref{fig:ularjpg} dengan skala 0.5 ditempatkan di antara dua buah paragraf. Perhatikan penulisannya di dalam file bab2.tex!
		\item Gambar~\ref{fig:ularpdf} ditempatkan menggunakan {\it specifier} {\bf h}.
	\end{itemize}
\end{itemize}
 
\dtext{17-18}
\begin{figure} 
	\centering  
	\includegraphics[scale=1]{ular-png}  
	\caption[Gambar {\it Serpentes} dalam format png]{Gambar {\it Serpentes} dalam format png} 
	\label{fig:ularpng} 
\end{figure} 

\dtext{19-20}
\begin{figure}[H]
	\centering  
	\includegraphics[scale=0.5]{ular-jpg}  
	\caption[Ular kecil]{Ular kecil} 
	\label{fig:ularjpg} 
\end{figure} 
\dtext{21-22}

\begin{figure}[ht] 
	\centering  
	\includegraphics[scale=1]{ular-pdf}  
	\caption[ {\it Serpentes} betina]{ {\it Serpentes} jantan} 
	\label{fig:ularpdf} 
\end{figure} 
 
