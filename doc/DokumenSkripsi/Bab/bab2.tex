%versi 2 (8-10-2016)
\chapter{Landasan Teori}
\label{chap:teori}
\section{Sistem Rekomendasi}
\label{sec:sistem rekomendasi}

Sistem rekomendasi adalah alat dan teknik perangkat lunak yang menyediakan saran untuk item yang akan digunakan oleh pengguna. Saran terkait dengan berbagai proses pengambilan keputusan, seperti barang apa yang akan dibeli, musik apa yang akan didengarkan, atau berita online apa yang akan dibaca.\\ % RS hanbook

Sistem rekomendasi biasanya berfokus pada item tertentu seperti buku,musik,dll. Sistem rekomendasi ditujukan untuk individu atau personal yang kurang memiliki pengalaman pribadi. Contoh Sistem rekomendasi buku adalah \textit{website} Amazon.com. Item yang ditawarkan sebagai daftar item peringkat. Sistem rekomendasi mencoba memprediksi produk dengan cara mengumpulkan referensi dari pengguna lainnya.\\ %RS hanbook

Pengembangan sistem rekomendasi dimulai dari pengamatan yang sederhana berupa rekomendasi yang diberikan oleh orang lain dalam membuat keputusan rutin sehari-hari bisa berupa buku, musik, film, rekrutmen karyawan, dll.\\ %RS hanbook

Sistem rekomendasi menghasilkan rekomendasi menggunakan berbagai jenis pengetahuan dan data tentang pengguna, item yang tersedia, dan transaksi sebelumnya, contohnya berupa e-commerce yang mengatasi masalah kelebihan informasi yang terjadi akibat transaksi pengguna sebelumnya. \\ %RS hanbook

% kemungkinan besar dihapus (ga kepake di skripsi, cuman informasi aja)
\subsection{Fungsi Sistem Rekomendasi}
Fungsi utama sistem rekomendasi adalah menemukan item yang relevan dengan kebutuhan pengguna.Selain untuk menemukan item yang relevan terdapat juga beberapa fungsing sistem rekomendasi, yaitu  : %RS hanbook
	\begin{enumerate}
		\item Meningkatkan jumlah penjualan barang\\
			Salah satu fungsi penting untuk sistem rekomendasi yang komersil. Peningkatan jumlah penjualan item ini disebabkan karena penjualan item dilakukan tepat sasaran kepada pembeli yang memang membutuhkan dan menginginkan item tersebut. Merekomendasikan item yang sesuai dengan kebutuhan atau minat pengguna.
			
		\item Menjual barang-barang yang lebih beragam\\
			Memberikan rekomendasi item yang mungkin sulit ditemukan oleh pengguna jika tanpa menggunakan sistem rekomendasi.
			
		\item Meningkatkan kepuasan pengguna\\
			Sistem rekomendasi yang dirancang dengan baik memberikan rekomendasi yang sesuai dengan kebutuhan pengguna sehingga pengguna akan merasa senang menggunakan sistem tersebut.
		
		\item Meningkatkan kesetiaan pengguna\\
			Pengguna akan tetap menggunakan sebuah \textit{website} jika sistem rekomendasi yang hasilkan rekomendasi yang sesuai dengan kebutuhan pengguna. 
			
		\item Lebih mengerti apa yang diinginkan pengguna\\
			Sistem dapat memebrikan hasil rekomendasi item yang sesuai dengan kebutuhan pengguna.
	\end{enumerate} \leavevmode

\subsection{Sumber Data dan Pengetahuan}
\label{sec:sumber data dan pengetahuan}
Sistem rekomendasi adalah sistem pemrosesan informasi yang secara aktif mengumpulkan berbagai jenis data untuk membangun rekomendasinya. Data utama berupa data item yang disarankan dan pengguna yang akan menerima rekomendasi. Data yang digunakan sistem rekomendasi mencakup pada tige jenis objek, yaitu : %RS hanbook
	\begin{enumerate}
	\item Item\\
		Item adalah objek yang direkomendasikan, item bisa ditandai oleh kompleksitasnya dan nilai atau kegunaannya. Bisa bernilai positif jika sesuai atau negatif jika tidak sesuai.
	
	\item Pengguna\\
		Pengguna adalah objek yang menggunakan sistem, memiliki tujuan dan karakteristik beragam. Pengguna juga dapat dijelaskan oleh data pola perilaku (pola penelusuran web, atau pola pencarian perjalanan)
	
	\item Transaksi\\
		Interaksi yang direkam antara pengguna sistem rekomendasi. Transaksi adalah data seperti log yang menyimpan informasi penting yang dihasilkan selama interaksi manusia-komputer dan berguna untuk algoritma pembuatan rekomendasi yang digunakan sistem. Bentuk dari peringkat yang populer di sistem rekomendasi :
		
		\begin{itemize}
			\item Peringkat numerik 1 - 5
		
			\item Peringkat ordinal (sangat setuju, setuju, netral, tidak setuju, dan sangat tidak setuju)
		
			\item Peringkat biner, buruk (0) dan baik (1)
		
			\item Peringkat unary menunjukkan bahwa pengguna telah mengamati atau membeli barang atau menilai barang secara positif
		\end{itemize}
		
	\end{enumerate}
	
\subsection{Teknik Rekomendasi}
\label{teknik rekomendasi}
Berikut adalah teknik-teknik yang dapat digunakan pada sistem rekomendasi : \\%RS hanbook
\begin{enumerate}
	\item \textit{Content-based}\\
		Sistem merekomendasikan item yang mirip berdasarkan item yang disukai pengguna di masa lalu. Kesamaan dihitung berdasarkan fitur(atribut) yang terkait dengan item. misal , review positif film komedi, maka akan direkomendasikan film di genre yang sama. 

	\item \textit{Collaborative Filtering} \\
		 Rekomendasi berdasarkan item yang disukai pengguna lain yang memiliki kesamaan. Implementasi paling sederhana, merekomendasikan item yang disukai pengguna lain dengan selera serupa di masa lalu.  \textit{Collaborative Filtering} populer dan banyak digunakan pada sistem rekomendasi. \textit{Nearest neighbors} meningkatkan popularitas karena sederhana, efisien, dan kemampuan mereka untuk menghasilkan rekomendasi yang akurat dan menunjukkan ciri personal tertentu.
	
	\item \textit{Demographic} \\
		Rekomendasi berdasarkan profil demografis pengguna. Asumsinya bahwa rekomendasi yang berbeda harus dihasilkan untuk demografis yang berbeda. Misalnya diarahkan ke web dengan bahasa atau negara pengguna. 

	\item \textit{Knowledge-based} \\
		Merekomendasikan item berdasarkan pengetahuan domain spesifik tentang fitur (atribut) item tertentu yang memenuhi kebutuhan atau referensi pengguna. 

	\item \textit{Community-based} \\
		Merekomendasikan item berdasarkan teman-teman pengguna. Bukti menunjukan bahwa orang cenderung lebih mengandalkan rekomendasi dari teman-teman dari pada rekomendasi dari orang yang belum dikenal. 

	\item \textit{Hybrid recomender systems} \\
		Kombinasi dari beberapa teknik yang sudah disebutkan sebelumnya. Menggunakan teknik A dan B mencoba untuk menggunakan keunggulan A dan memperbaiki kelemahan B. Contoh, \textit{Collaborative Filtering} memiliki kelemahan terhadap item yang tidak memiliki peringkat (tidak terdapat riwayat) bisa digabungkan dengan metode \textit{Content-based}.
\end{enumerate}


% jelasin konsep dasar Collabrative Filtering
\subsection{\textit{Collaborative Filtering}}
\label{sec:collaborative filtering}
	Dalam pengembangan sistem rekomendasi dapat menggunkan teknik \textit{Collaborative Filtering}. \textit{Collaborative Filtering} menghasilkan rekomendasi item yang spesifik untuk pengguna berdasarkan peringkat tanpa memerlukan informasi tambahan mengenai item ataupun pengguna. Gagasan utamanya adalah peringkat pengguna u untuk item i cenderung mirip dengan pengguna v, jika u dan v memberikan peringkat item lain dengan nilai yang sama. \\%RS hanbook
	Tantangan dalam membangun sistem rekomendasi menggunakan teknik \textit{Collaborative Filtering} adalah sedikitnya jumlah data pengguna sebelumnya yang sudah memberikan peringkat kepada suatu item. Dalam \textit{Collaborative Filltering} terdapat salah satu algoritma yaitu \textit{Neighborhood-based Collaborative Filtering} atau yang dikenal dengan \textit{Memory-base Collaborative Filtering}.
	
\subsubsection{\textit{Neighborhood-based Collaborative Filtering}}
\textit{Neighborhood-based Collaborative Filtering} atau yang dikenal dengan \textit{Memory-base Collaborative Filtering} adalah algoritma pertama yang dikembangan untuk teknik \textit{Collaborative Filtering}. Pada algoritma ini \textit{rating} \textit{user-item} disimpan dalam sistem secara langsung digunakan untuk memprediksi peringkat item baru, dapat dilakukan dengan \textit{user-based model}. %RS hanbook

%berdasarkan RS Book 4.2.1
\subsubsection{\textit{User-based Neighborhood Model}}
\label{user-based}
\textit{User-based} bekerja dengan mengidentifikasi pengguna yang akan diberikan rekomendasi dengan pengguna lain yang memiliki kesamaan. Aktivitas pengguna yang memiliki kesamaan ini akan menjadi dasar dalam memberikan rekomendasi kepada pengguna lain. Aktivitas bisa berupa memberikan \textit{rating} kepada item. Berikut adalah tahapan yang perlu dilakukan pada \textit{User-based Neighborhood Model} : 

\begin{enumerate}
	%dilakukkan pada saat preprosesing	
	\item Menghitung nilai rata-rata \textit{rating} yang sudah diberikan oleh pengguna lain.
	
	\item Menghitung kesamaan atau similaritas pengguna menggunakan \textit{Pearson Correlation Coefficient} 2.1 :
	
	\begin{equation}
		sim(i,j) = Pearson(i,j) = \frac{\Sigma _{k\epsilon} I_{i} \cap I_{j} (r_{i,k}-\mu_{i}) \cdot (r_{j,k}-\mu_{j})}{\sqrt{\Sigma _{k\epsilon} I_{i} \cap I_{j} (r_{i,k}-\mu_{i})^2} \cdot \sqrt{\Sigma _{k\epsilon} I_{i} \cap I_{j} (r_{j,k}-\mu_{j})^2 }}
	\end{equation}\leavevmode \\
	Keterangan : 
	\begin{itemize}
		\item sim(i,j) = Kesamaan atau similaritas antara pengguna i dan pengguna j
		
		\item $\Sigma _{k\epsilon} I_{i} \cap I_{j}$ = Himpunan item pengguna i dan pengguna j yang saling beririsan
		
		\item $r_{i,k}$ = Nilai yang diberikan pengguna i terhadap item k
		
		\item $r_{j,k}$ = Nilai yang diberikan pengguna j terhadap item k
		
		\item $\mu_{i}$ = Rata-rata nilai yang diberikan pengguna i
		
		\item $\mu_{j}$ = Rata-rata nilai yang diberikan pengguna j
	\end{itemize}\leavevmode
	
	\item Memilih nilai kesamaan atau similaritas yang bernilai lebih besar dari 0. Nilai keasamaan atau similaritas memiliki rentan nilai -1, 0, dan +1 untuk pengguna yang akan diberikan rekomendasi. Jika hasil perhitungan mendekati -1, berarti pengguna tersebut kurang memiliki kesamaan dengan pengguna yang akan diberikan rekomendasi. Jika hasil perhitungan mendekati 0, berarti pengguna tersebut memiliki kesamaan yang cukup baik dengan pengguna yang akan diberikan rekomendasi. Jika hasil perhitungan mendekati +1, berarti pengguna tersebut memiliki kesamaan yang tinggi dengan pengguna yang akan diberikan rekomendasi.
	%kayanya bagian ini ga usah, soalnya nnti ambil K tetangga terdekat, liat IPK + jurusan, filteri IPK > 2, trs count jurusannya, rekomendasiin
	\item Menghitung nilai prediksi dengan rumus \textit{weighted sum} 2.2 :
	
	%\begin{equation}
	%	r_{i,k} = \mu_{i} + \frac{\Sigma _{j \epsilon} P_{i} Sim(i,j)\cdot (r_{j,k} - \mu_{j})}{\Sigma _{j \epsilon} P_{i(k)} |Sim(i,j)|}
	%\end{equation}\leavevmode \\

	\begin{equation}
		r_{i,k} = \frac{\Sigma (Sim(i,j)*r_{j,k}) }{\Sigma Sim(i,j)}
	\end{equation}		
		
	Keterangan :
	\begin{itemize}
		\item $r_{i,k}$ = Nilai prediksi pengguna i untuk item k
		
		\item Sim(i,j)= kesamaan atau similiaritas pengguna i dan pengguna j
		
		\item $r_{j,k} $ = Penilaian pengguna j terhadap item k
		
		%\item $r_{j,k}$ = Penilaian pengguna j terhdap item k
		
		%\item $\mu_{j}$ = Nilai rata-rata pengguna j
		
		%\item $\mu_{i}$ = Nilai rata-rata pengguna i
	\end{itemize}\leavevmode
	
	\item Mengurutkan nilai prediksi dari yang terbesar ke terkecil. %[3]
	 
\end{enumerate}\leavevmode

\subsection{Aplikasi dan Evalusi}
\label{sec:aplikasi dan evaluasi}
\subsubsection{Aplikasi}
Faktor pertama yang harus dipertimbangkan adalah domain aplikasi yang akan dibangun karena memiliki efek yang besar pada algoritma yang akan digunakan. Kelas domain paling umum : %RS hanbook
\begin{enumerate}
	\item Entertainment : rekomendasi film dan musik
	\item Content : personalisasi berita, dokukumen, dan web page
	\item E-commerce : rekomendasi produk untuk di beli
	\item Services : rekomendasi servis travel, hotel, dan rumah
\end{enumerate}

\subsubsection{Evalusi}
Sebuah sistem rekomendasi banyak digunakan untuk memberikan prediksi berupa saran item yang sesuai dengan minat pengguna. Prediksi yang diberikan sistem rekomendasi memiliki nilai keakuratan yang dapat berbeda sesuai dengan kasus yang dihadapi dan juga algoritma yang digunakan. Prediksi yang diberikan harus akurat, oleh karena itu diperlukan evaluasi pada sistem rekomndasi. Evalusasi dapat menggunakan tiga metode yaitu :

\begin{enumerate}
	\item \textit{Offline}\\
		Metode \textit{offline} dilakukan dengan cara menjalankan beberapa algoritma pada data yang sama dan membandingkan kinerjanya.

	\item \textit{Online}\\
		Metode \textit{online} dilakukan saat perangkat lunak sudah diluncurkan dan melibatkan pengguna nyata. 
		
	\item \textit{Focused user study}\\
		Metode \textit{Focused user study} dilakukkan saat metode \textit{online} tidak layak dilakukan atau terlalu beresiko.
	
\end{enumerate}

%Evaluasi diperlukan untuk sistem rekomendasi. Evaluasi \textit{offline} dilakukan dengan cara menjalankan beberapa algoritma pada data yang sama dan membandingkan kinerjanya. Evaluasi \textit{online} dilakukan saat sistem sudah diluncurkan, melibatkan pengguna nyata. \textit{Focused user study}, jika evaluasi online tidak layak atau terlalu beresiko, meminta beberapa pengguna untuk mencoba sistem.

\section{\textit{Cluster}}
\label{sec:cluster}
%penjelasan cluster (dari buku hashrul)
\textit{Clustering} adalah algoritma yang menganalisis objek data tanpa perlu label kelas. \textit{Clustering} bisa digunakan untuk menghasilkan label kelas untuk sebuah kelompok data. Tujuan dari algoritma \textit{cluster} adalah untuk meminimalkan jarak \textit{intra-cluster} sekaligus memaksimalkan jarak \textit{inter-cluster}. Kesamaan ditentukan dengan menggunakan ukuran jarak. Kelompok yang dihasilkan akan memiliki anggota yang memiliki kesamaan yang tinggi satu sama lain  didalam kelompok yang sama dan berbeda dengan kelompok lain.

\subsection{K-Means}
% penjelasan k means (dari buku RS)
K-Means adalah algoritma \textit{clustering} yang termasuk kategori \textit{partisial}. Kategori partisial adalah membagi item kedalam \textit{non-overlapping cluster} sehingga setiap item hanya ada pada satu \textit{cluster}. \textit{Cluster} akan dibentuk sebanyak K dan untuk setiap \textit{cluster} memiliki \textit{centroid} awal. \textit{Centroid} untuk setiap \textit{cluster} adalah titik dimana jumlah jarak minimum dari semua item dalam \textit{cluster}. Berikut merupakan tahapan-tahapan dalam pembutan \textit{cluster} :
\begin{enumerate}
	\item Memilih secara acak K \textit{centroid}
	
	\item Menghitung jarak tiap objek ke titik \textit{centroid} menggunakan \textit{euclidean distance}
	
	\item Menghitung \textit{centroid} baru dari anggota yang berada didalam \textit{centroid}
	
	\item Lakukan tahap 1-3 hingga \textit{centroid} konvergen
\end{enumerate} 

\section{\textit{Library} PHP-ML}
\label{sec:library php-ml}
%Penjelasan php ml
PHP-ML adalah sebuah \textit{library} yang khusus dibuat untuk \textit{Machine Learning} dengan menggunakan bahasa pemrograman PHP. Terdapat lebih dari 20 algoritma yang bisa digunakan. \textit{Library} ini bersifat \textit{open source} yang berlisensi MIT. Versi PHP minimal untuk menggunakan \textit{library} ini adalah PHP 7.1, pengingstallan dapat menggunakan Composer.

% https://php-ml.readthedocs.io/en/latest/machine-learning/datasets/array-dataset/
\subsection{Array Dataset} % perlu di italic atau engga
\textit{Array Dataset} adalah bagian dari fitur \textit{Dataset} yang disediakan oleh PHP-ML. \textit{Array Dataset} adalah kelas yang berfungsi untuk menyimoan data sebagai tipe array dalam PHP. Menerapkan \textit{interface} Dataset yang banyak digunakan di kelas lain. Kelas ini memiliki dua parameter yaitu : \textit{samples} dan \textit{labels}. \textit{Samples} adalah array yang berisikan sample. \textit{Labels} adalah array yang berisikan \textit{label} setiap \textit{sample}.

% https://php-ml.readthedocs.io/en/latest/machine-learning/cross-validation/random-split/
\subsection{Random Split}
\textit{Random Split} adalah bagian dari fitur \textit{Cross Validation} yang disediakan oleh PHP-ML. Kelas \textit{Random Split} adalah salah satu metode paling sederhana dari \textit{Cross Validation}. \textit{Samples} dibagi menjadi dua kelompok yaitu : \textit{train group} dan \textit{test group}. Kelas ini memiliki tiga parameter yaitu : \textit{dataset}, \textit{testSize}, dan \textit{seed}. \textit{Dataset} adalah objek yang mengimplementasikan \textit{interface} Dataset. \textit{TestSize} adalah bilangan \textit{float} yang menyatakan seberapa banyak anggota pada \textit{test group} dengan nilai dasar 0.3 jika parameter tidak diisi. \textit{Seed} untuk \textit{random generator}.

\section{Universitas Katolik Parahyangan}

%\subsection{Universitas Katolik Parahyangan}
Perguruan tinggi adalah satuan pendidikan yang menyelenggarakan pendidikan tinggi yang dapat berbentuk akademi, politeknik, sekolah tinggi, institut. atau universitas. Pendidikan tinggi adalah kelanjutan pendidikan menengah yang diselenggarakan untuk menyiapkan peserta didik menjadi anggota masyarakat yang memiliki kemampuan akademik dan/atau profesional yang dapat menerapkan, mengembangkan dan/atau menciptakan ilmu pengetahuan. teknologi dan/atau kesenian. \\ %SK MENDIKNAS 232

Universitas Katolik Parahyangan adalah sebuah univessitas atau Perguruan tinggi katolik pertama yang didirikan pada 17 Januari 1955. Saat ini terletak di Jalan Ciumbuleuit No.94, Bandung, Jawa Barat, Indonesia. Terdapat tujuh fakultas dengan total program studi yaitu tujuh belas dengan enam belas program studi sarjana dan satu program studi D3. %unpar.ac.id

Terdapat beberapa jalur penerimaan mahasiswa baru yang dilakukan oleh Universitas Katolik Parahyangan. Jalur penerimaan diselenggarakan secara mandiri, berikut jalur penerimaan yang disediakan Universitas Katolik Parahyangan :

\begin{enumerate}
	\item Penelusuran Minat dan Kemampuan (PMKD) atau jalur prestasi\\
		PMDK adalah satu jalur penerimaan mahasiswa baru yang dilaksakan dengan seleksi berdasarkan pada nilai raport SMA di kelas X (Sepuluh) dan XI (Sebelas), tanpa ujian tertulis. Tujuan dari PMDK untuk menjaring siswa-siswa yang berprestasi. PMDK dilakukan hanya satu kali dalam satu tahun penerimaan.
		
	\item Ujian Saringan Masuk (USM)\\
		USM adalah satu jalur penerimaan mahasiswa baru yang dilaksanakan dengan mengerjakan soal yang disediakan oleh Universitas Katolik Parahyangan. Terdapat dua tempat pelaksanaan untuk USM, pertama dilaksakan di Universitas Katolik Parahyangan dan kedua dilaksakan di sekolah-sekolah (\textit{on-site test}.Tujuan dari USM untuk menjaring mahasiswa baru yang memiliki kemampuan akademik untuk mengikuti dan menyelesaikan pendidikan di Universitas Katolik Parahyangan sesuai dengan batas waktu (masa studi) yang ditetapkan.
	
\end{enumerate}

\subsection{Program Studi}
\label{sec:program studi} 
%Diisi dengan penjelasan program studi dan syarat-syaratnya
Program studi adalah kesatuan rencana belajar sebagai pedoman penyelenggaraan pendidikan akademik dan/atau profesional yang diselenggarakan atas dasar suatu kurikulum serta ditujukan agar mahasiswa dapat menguasai pengetahuan, keterampilan, dan sikap sesuai dengan sasaran kurikulum. Kurikulum pendidikan tinggi adalah seperangkat rencana dan pengaturan mengenai isi maupun bahan kajian dan pelajaran serta cara penyampaian dan penilaiannya yang digunakan sebagai pedoman penyelenggaraan kegiatan belajar - mengajar di perguruan tinggi. \\ %SK MENDIKNAS 232

Terdapat tujuh fakultas yang ada di Universitas Katolik Parahyangan, yaitu :
	\begin{enumerate}
		\item Fakultas Ekonomi 
		\item Fakultas Hukum
		\item Fakultas Ilmu Sosial dan Ilmu Politik
		\item Fakultas Teknik
		\item Fakultas Falsafah dan Peradaban
		\item Fakultas Teknologi Industri
		\item Fakultas Teknologi Informasi dan Sains
	\end{enumerate}\leavevmode
	
\subsubsection{Fakultas Ekonomi}
Terdapat empat program studi pada fakultas Ekonomi, yaitu : Ekonomi Pembangunan, Manajemen, Akuntansi, Manajemen Perusahaan. Manajemen Perusahaan merupakan program studi D3 yang ada di Universitas Katolik Parahyangan. Berikut merupakan penjelasan program studi yang ada pada Fakultas Ekonomi :
	
	\begin{enumerate}
		\item Ekonomi Pembangunan\\
			 Mempelajari persoalan pembangunan ekonomi yang sudah, sedang, dan akan
terjadi di negara berkembang. Menganalisis isu perekonomian untuk mencari dan menemukan solusi dari berbagai persoalan ekonomi secara kritis, kreatif, dan inovatif. Program studi Ekonomi Pembangunan mempersiapkan mahasiswanya untuk menjadi perencana bidang pembangunan ekonomi. Ekonomi Pembangunan adalah cabang ilmu ekonomi. Mempelajari pembangunan industri, perbankan, keuangan, dan bisnis. Berkutat dengan analisis berbagai isu perekonomian untuk mendapatkan solusi dari persoalan ekonomi.\\

			Terdapat tiga peminatan pada program studi Ekonomi Pembangunan, yaitu :
			\begin{itemize}
				\item Ekomoni Industri dan Perdagangan
				\item Ekonomi Kawasan dan Lingkungan
				\item Ekonomi Moneter dan Keuangan
			\end{itemize}\leavevmode

		\item Manajemen\\
			Mempelajari bagaimana mengelola suatu perusahaan atau organisasi. Fokus pada kegiatan mengelola, merencanakan, dan mengatur semua proses dalam perusahaan untuk mencapai tujuan.\\
			
			Terdapat satu peminatan pada program studi Manajemen, yaitu :
			\begin{itemize}
				\item Manajemen
			\end{itemize}\leavevmode

		\item Akuntansi\\
			Mempelajari mengenai keuangan dan ilmu ekonomi, Mahasiswa pada program studi Akuntansi akan memiliki pengetahuan dan penguasaan materi tentang keuangan dan ilmu ekonomi. Mampu mengelola keuangan bisnis.\\
			
			Terdapat satu peminatan pada program studi Akuntansi, yaitu :
			\begin{itemize}
				\item Akuntansi
			\end{itemize}\leavevmode
		
		%\item Manajemen Perusahaan\\
			%Terdapat satu peminatan pada program studi Manajemen Perusahaan, yaitu :
			%\begin{itemize}
				%\item Manajemen Perusahaan
			%\end{itemize}\leavevmode
			
	\end{enumerate}\leavevmode

\subsubsection{Fakultas Hukum}
Terdapat satu program studi pada Fakultas Hukum, yaitu : Ilmu Hukum.

	%program studi
	\begin{enumerate}
		\item Ilmu Hukum\\
			Mempelajari tentang hukum baik praktek maupun teori. Hukum mengatur bagaimana manusia bertindak dan bertingkah laku agar tidak merugikan orang lain. Mendalami konsep, teori, dan beberapa kasus hukum yang terjadi.\\
			
			Terdapat satu peminatan pada program studi Ilmu Hukum, yaitu :
			
			\begin{itemize}
				\item Ilmu Hukum
			\end{itemize}\leavevmode
			
	\end{enumerate}\leavevmode

\subsubsection{Fakultas Ilmu Sosial dan Ilmu Politik}
Terdapat tiga program studi pada Fakultas Ilmu Sosial dan Ilmu Politik, yaitu : Ilmu Administrasi Publik, Ilmu Administrasi Bisni, dan Ilmu Hubungan Internasional.
	%program studi
	\begin{enumerate}
		\item Ilmu Administrasi Publik\\
			Mempelajari seluk beluk pemerintahan, masyarakat, dan kebijakan
publik, sistem pemerintahan, pembuatan kebijakan hingga pengimplementasian dan evaluasi, pelayanan masyarakat, dan segala sesuatu yang berkaitan dengan birokrasi.\\

			Terdapat satu peminatan pada program studi Ilmu Administrasi Publik, yaitu :
			
			\begin{itemize}
				\item Ilmu Administrasi Publik
			\end{itemize}\leavevmode
			
		\item Ilmu Administrasi Bisni\\
			Mempelajari mengenai kegiatan operasional bisnis dan perusahaan,
yaitu : pemasaran (marketing), pengelolaan keuangan, pengelolaan personalia (SDM), hingga kegiatan produksi. Mempelajari untuk membuat produk sendiri, bukan membuat, menjual, dan mendapatkan keuntungan, tetapi menciptakan value pada produk yang dipasarkan. Mempelajari urusan klarikal kantor, mengelola sarana dan prasarana kantor, memproses data secara akurat, dan mengelola informasi yang berhubungan dengan pekerjaan kantor. Program studi ini cocok dengan orang yang memiliki ketertarikan dalam bidang pengurusan dokumen.\\
			
			Terdapat dua peminatan pada program studi Ilmu Administrasi Bisnis, yaitu :
			
			\begin{itemize}
				\item \textit{General Business}
				\item \textit{Digital Business}
			\end{itemize}\leavevmode
						
		\item Ilmu Hubungan Internasional\\
			Mempelajari mengenai interaksi, relasi, dan komunikasi yang terjadi secara internasional. Tidak hanya mempelajari hubungan diplomasi satu negara dengan negara lain, tapi juga konflik, kesejahteraan, ekonomi, dan perdamaian dunia. Beberapa kajian diplomasi dan negosiasi, politik luar negeri, perdagangan luar negeri, politik internasional, ekonomi internasional, hukum internasional, globalisasi, dll. Diasah mengenai isu-isu global, tokoh-tokoh, dan organisasi internasional yang berpengaruh, dan kerjasama internasional.\\
			
			Terdapat satu peminatan pada program studi Ilmu Administrasi Bisnis, yaitu :
			
			\begin{itemize}
				\item Ilmu Hubungan Internasional
			\end{itemize}\leavevmode

	\end{enumerate}\leavevmode
	
\subsubsection{Fakultas Teknik}
Terdapat dua program studi pada Fakultas Teknik, yaitu : Teknik Sipil dan Arsitektur.
	%program studi
	\begin{enumerate}
		\item Teknik Sipil\\
			 Mempelajari proses merancang, membangun, dan merenovasi gedung serta infrastruktur lain, seperti jalan, jembatan, bendungan, dan infrastruktur lainnya. Memahami unsur-unsur bangunan seperti beton, baja, aspal, dan lain-lain. Mempelajari perancangan struktur bangunan yang kuat, layak, dan efisien.\\
			
			Terdapat satu peminatan pada program studi Teknik Sipil, yaitu :
			
			\begin{itemize}
				\item Teknik Sipil
			\end{itemize}\leavevmode
			
		\item Arsitektur\\
			Mempelajari desain dan rancangan konstruksi bangunan. Lebih menuangkan ide, konsep, dan desain di atas kertas, sedangkan realisasi akan dikerjakan oleh teknik sipil. Harus mempelajari kekuatan bangunan (firmitasi), estetika atau keindahan bangunan (venustas), dan fungsi bangunan (utilitas). \\
			
			Terdapat satu peminatan pada program studi Arsitektur, yaitu :
			
			\begin{itemize}
				\item Arsitektur
			\end{itemize}\leavevmode

	\end{enumerate}\leavevmode
	
\subsubsection{Fakultas Falsafah dan Peradaban}
Terdapat satu program studi pada Fakultas Falsafah dan Peradaban, yaitu : Ilmu Filsafat.
	
	%program studi
	\begin{enumerate}
		\item Ilmu Filsafat\\
			 Filsafat sebagai induk semua ilmu, filsafat lebih mempelajari tentang permasalahan mendasar manusia dan hubungannya dengan realita. Bersifat abstrak dan memerlukan pemahaman yang mendasar. Kajian utamanya yaitu tujuan hidup, esensi manusia, moralitas, dan hati nurani. Mempelajari pemikiran para filsuf. Membantu berpikir secara terstruktur dan mampu memproses informasi secara jernih.\\
			
			Terdapat dua peminatan pada program studi Ilmu Filsafat, yaitu :
			
			\begin{itemize}
				\item Filsafat Keilahian
				\item Filsafat Budaya
			\end{itemize}\leavevmode
			
	\end{enumerate}\leavevmode
	
\subsubsection{Fakultas Teknologi Industri}
Terdapat tiga program studi pada Fakultas Teknologi Industri, yaitu : Teknik Industri, Teknik Kimia, dan Teknik Elektro.
	%program studi
	\begin{enumerate}
		\item Teknik Industri\\
			Mempelajari proses industri baik dari sisi manajemen ataupun teknik. Turunan dari teknik mesin. Mempelajari disiplin ilmu lain seperti matematika, fisika, fisiologi, dan manajemen saintifik. Teknik Industri berfokus pada perancangan, peningkatan, dan pemasangan sistem terintegrasi yang membutuhkan manusia, material, peralatan, dan energi. Memiliki tiga bidang dan satu sistem manufaktur (mempelajari peningkatan kualitas, produktivitas, dan efisiensi sistem produk), dua manajemen industri (mempelajari manajemen keuangan, operasional, manajemen inovasi, perencanaan dan pengendalian produksi, dan ekonomi teknik), dan tiga sistem industri dan tekno ekonomi, seperti logistik, statistik, penelitian operasional, dan sistem basis data.\\
			
			Terdapat satu peminatan pada program studi Teknik Industri, yaitu :
			
			\begin{itemize}
				\item Teknik Industri
			\end{itemize}\leavevmode
			
		\item Teknik Kimia\\
			Cabang ilmu teknik yang mempelajari bagaimana proses dan cara mengubah bahan baku/mentah dan bahan kimia menjadi sebuah produk yang lebih bernilai secara komersial maupun perubahan sifat fisik dan kimia bahan mentah. Dididik untuk merencanakan dan merancang alat-alat proses, mengoperasikan, mengendalikan dan memelihara pabrik/industri, mengkontruksi pendirian suatu pabrik, mengadakan penelitian dan pengembangan proses, serta merencanakan serta mengelola penjualan dan pelayanan.\\
			
			Terdapat satu peminatan pada program studi Teknik Kimia, yaitu :
			
			\begin{itemize}
				\item Teknik Kimia
			\end{itemize}\leavevmode

		\item Teknik Elektro\\
			Mempelajari sifat-sifat elektron yang kita kenal sebagai listrik, mempelajari aplikasi dan pemanfaatan listrik dalam kehidupan sehari-hari, serta teknologi yang terkait. Cakupannya meliputi pembangkit tenaga listrik, sistem jaringan distribusi, pemanfaatan oleh pengguna akhir.\\
			
			Terdapat satu peminatan pada program studi Teknik Elektro, yaitu :
			
			\begin{itemize}
				\item Mekatronika
			\end{itemize}\leavevmode

	\end{enumerate}
	
\subsubsection{Fakultas Teknologi Informasi dan Sains}
Terdapat tiga program studi pada Fakultas Teknologi Informasi dan Sains, yaitu : Matematika, Fisika, dan Teknik Informatika.
	
	%program studi
	\begin{enumerate}
		\item Matematika\\
			Mempelajari matematika murni seperti aljabar, geometri, dan analisis matematika; statistika; komputasi; aktuaria; dan riset operasi.\\

			Terdapat dua peminatan pada program studi Matematika, yaitu :
			
			\begin{itemize}
				\item Aktuaria
				\item Matematika Terapan
			\end{itemize}\leavevmode
			
		\item Fisika\\
			 Mempelajari gejala alam yang tidak hidup atau materi dalam lingkup ruang dan waktu, mempelajari perilaku dan sifat materi dalam bidang yang sangat beragam (partikel submikroskopis - perilaku materi alam semesta sebagai satu kesatuan kosmos). Ilmu fisika sangat mendukung perkembangan teknologi, yaitu industri, komunikasi, kerekayasaan, kimia, dan kedokteran.\\
			 
			 Terdapat satu peminatan pada program studi Fisika, yaitu :
			
			\begin{itemize}
				\item Fisika
			\end{itemize}\leavevmode

		\item Teknik Informatika\\
			Mempelajari dan menerapkan prinsip-prinsip ilmu komputer dan analisa matematis untuk desain, pengembangan, pengujian, evaluasi perangkat lunak, sistem operasi, dan kerja komputer. Menghasilkan ide kreatif, merealisasikan ide, mendiferensiasikan berbagai macam fungsi, dan menciptakan struktur instruksi yang sangat detail dalam bahasa pemrograman untuk mengajarkan komputer apa yang harus dilakukan.\\
			
			Terdapat dua peminatan pada program studi Teknik Informatika, yaitu :
			
			\begin{itemize}
				\item \textit{Data Science}
				\item \textit{Computer Science}
			\end{itemize}\leavevmode
			
	\end{enumerate}\leavevmode
	
\subsection{Syarat Masuk Program Studi}
Berikut merupakan syarat untuk program studi yang ada di Universitas Katolik Parahyangan :
\begin{longtable}[H]{|p{3cm}|p{2cm}|p{3cm}|p{3cm}|p{3cm}|} %atau h saja untuk "kira kira di sini"
	%\centering 

	%\begin{tabular}{|p{3cm}|p{2cm}|p{3cm}|p{3cm}|p{3cm}|}
		\hline
		Program Studi & Syarat Jurusan & USM  & PMDK & Syarat Khusus\\

		\hline
		\multirow{3}{10em}{Ekonomi Pembangunan} & IPA & Matematika & Matematika & \\
		& IPS & Bahasa Inggris & Bahasa Indonesia & \\
		& & & Bahasa Inggris & \\
		
		\hline
		\multirow{2}{10em}{Manajemen} & IPA & Matematika & Matematika & \\
		& IPS & Bahasa Inggris & Bahasa Inggris & \\
		
		\hline
		\multirow{2}{10em}{Akuntansi} & IPA & Matematika & Matematika & \\
		& IPS & Bahasa Inggris & Bahasa Inggris & \\
		
		\hline
		%\multirow{4}{10em}{Manajemen Perusahaan} & IPA & Esai & Matematika & \\
		%& IPS & & & \\
		%& Bahasa & & & \\
		%& SMK & & & \\
		
		\hline
		\multirow{3}{10em}{Ilmu Hukum} & IPA & Matematika & Matematika & \\
		& IPS & Bahasa Inggris & Bahasa Inggris & \\
		& Bahasa & & Pendidikan Kewarganegaraan & \\
		
		\hline
		\multirow{4}{10em}{Ilmu Administrasi Publik} & IPA & Matematika & Matematika & \\
		& IPS & Bahasa Inggris & Bahasa Inggris & \\
		& Bahasa & & & \\
		& SMK & & & \\
		
		\hline
		\multirow{4}{10em}{Ilmu Administrasi Bisnis} & IPA & Matematika & Matematika & \\
		& IPS & Bahasa Inggris & Bahasa Inggris & \\
		& Bahasa & & & \\
		& SMK & & & \\
		
		\hline
		\multirow{3}{10em}{Ilmu Hubungan Internasional} & IPA & Matematika & Matematika & \\
		& IPS & Bahasa Inggris & Bahasa Inggris & \\
		& Bahasa & & Uraian Bahasa Inggris & \\
		
		\hline
		\multirow{3}{10em}{Teknik Sipil} & IPA & Matematika & Matematika & \\
		& & Bahasa Inggris & Bahasa Inggris & \\
		& & Fisika & Fisika & \\
		
		\hline
		\multirow{3}{10em}{Arsitektur} & IPA & Matematika & Matematika & \\
		& & Bahasa Inggris & Bahasa Inggris & \\
		& & Gambar & Gambar & \\
		
		\hline
		\multirow{4}{10em}{Ilmu Filsafat} & IPA & Matematika & Bahasa Inggris & \\
		& IPS & Bahasa Inggris & Bahasa Indonesia & \\
		& Bahasa & Wawancara & & \\
		& SMK & & & \\
		
		\hline
		\multirow{2}{10em}{Teknik Industri} & IPA & Matematika & Bahasa Inggris & \\
		& & Matematika & Bahasa Inggris & \\
		
		\hline
		\multirow{4}{10em}{Teknik Kimia} & IPA & Matematika & Matematika & Tidak buta warna\\
		& & Bahasa Inggris & Bahasa Inggris & \\
		& & Fisika & Fisika & \\
		& & & Kimia & \\
		
		\hline
		\multirow{3}{10em}{Teknik Elektro} & IPA & Matematika & Matematika & Tidak buta warna \\
		& & Bahasa Inggris & Bahasa Inggris & \\
		& & Fisika & Fisika & \\
		
		\hline
		\multirow{2}{10em}{Matematika} & IPA & Matematika & Matematika & \\
		& & Bahasa Inggris & Bahasa Inggris & \\
		& & & & \\
		
		\hline
		\multirow{3}{10em}{Fisika} & IPA & Matematika & Matematika & \\
		& & Bahasa Inggris & Bahasa Inggris & \\
		& & Fisika & & \\
		
		\hline
		\multirow{2}{10em}{Teknik Informatika} & IPA & Matematika & Matematika & \\
		& & Bahasa Inggris & Bahasa Inggris & \\
		
		\hline
	%\end{tabular} 
	\caption{Tabel syarat program studi}
	\label{tab:tabel syarat program studi}
\end{longtable}

\subsection{Karakteristik Program Studi}
Berikut merupakan kriteria untuk calon mahasiswa sesuai dengan program studi :



\begin{longtable}[H]{| p{6cm} | p{8cm} |} %atau h saja untuk "kira kira di sini"
	%\centering 
	%\begin{tabular}
		\hline
		Program Studi & Karakteristik \\

		\hline
		\multirow{5}{10em}{Ekonomi Pebangunan} & Tertarik dengan Ilmu Ekonomi \\
		& Tertarik dengan perhitungan \\
		& Berpikir kritis \\
		& Senang menganalisis\\
		& Mampu memecahkan masalah\\
		
		\hline
		\multirow{3}{10em}{Manajemen} & Keterampilan komunikasi \\
		& Senang menganalisis\\
		& Senang memecahkan masalah\\
		
		\hline
		\multirow{3}{10em}{Akuntansi} & Tertarik dengan akuntansi \\
		& Memiliki kemampuan berhitung yang kuat dan teliti \\
		& Senang menganalisis\\
		
		\hline
		%\multirow{3}{10em}{Manajemen Perusahaan} &  \\
		%& \\
		%& \\
		
		\hline
		\multirow{4}{10em}{Ilmu Hukum} & Tertarik dengan hukum \\
		& Teliti dan berpikir kritis \\
		& Keterampilan komunikasi \\
		& Mampuan menganalisis\\
		
		\hline
		\multirow{3}{10em}{Ilmu Administrasi Publik} & Terstruktur \\
		& Senang menganalisis\\
		& Senang memecahkan masalah\\
		
		\hline
		\multirow{4}{10em}{Ilmu Administrasi Bisnis} & Memiliki minat yang tinggi untuk usaha \\
		& Kemampuan komunikasi\\
		& Kemampuan berhitung\\
		& Terstruktur\\
		
		\hline
		\multirow{4}{10em}{Ilmu Hubungan Internasional} & Tertarik dengan interaksi internasional \\
		& Kemampuan berbahasa Inggris \\
		& Berwawasan luas \\
		& Kemampuan komunikasi\\
		
		\hline
		\multirow{2}{10em}{Teknik Sipil} & Senang berhitung \\
		& Terstruktur \\
		
		\hline
		\multirow{4}{10em}{Arsitektur} & Tertarik dengan desain dan rancangan bangunan \\
		& Tertarik dengan menggambar dan seni\\
		& Tertarik dengan humaniora, sains, dan teknologi \\
		
		\hline
		\multirow{4}{10em}{Ilmu Filsafat} & Tipe pemikir \\
		& Berwawasan luas \\
		& Berpikir Rasional \\
		& Berpikir kritis \\
		
		\hline
		\multirow{2}{10em}{Teknik Industri} &  Senang berhitung \\
		& Terstruktur \\
		
		\hline
		\multirow{4}{10em}{Teknik Kimia} & Tertarik dengan Kimia \\
		& Senang berhitung \\
		& Terstruktur \\
		& Tidak buta warna\\
		
		\hline
		\multirow{4}{10em}{Teknik Elektro} & Tidak buta warna \\
		& Senang berhitung \\
		& Terstruktur \\
		& Teliti \\

		\hline
		\multirow{4}{10em}{Matematika} & Tertarik dengan Matematika \\
		& Senang memecahkan masalah \\
		& Terstruktur \\
		& Teliti \\
		
		\hline
		\multirow{4}{10em}{Fisika} & Senang berhitung \\
		& Senang menganalisis \\
		& Mampu memecahkan masalah\\
		& Teliti \\
		
		\hline
		\multirow{4}{10em}{Teknik Informatika} & Tertarik dengan teknologi \\
		& Senang menganalisis \\
		& Senang memecahkan masalah \\
		& Senang berhitung \\
		
		\hline
	%\end{tabular} 
	\caption{Tabel kriteria}
	\label{tab:tabel kriteri}
\end{longtable}


\section{Template Skripsi FTIS UNPAR}
\label{sec:template}
 
Akan dipaparkan bagaimana menggunakan template ini, termasuk petunjuk singkat membuat referensi, gambar dan tabel.
Juga hal-hal lain yang belum terpikir sampai saat ini. 
 
\dtext{15-16}

\subsection{Tabel}  
Berikut adalah contoh pembuatan tabel. 
Penempatan tabel dan gambar secara umum diatur secara otomatis oleh \LaTeX{}, perhatikan contoh di file bab2.tex untuk melihat bagaimana cara memaksa tabel ditempatkan sesuai keinginan kita.

Perhatikan bawa berbeda dengan penempatan judul gambar gambar, keterangan tabel harus diletakkan di atas tabel!!
Lihat Tabel~\ref{tab:contoh1} berikut ini:

\begin{table}[H] %atau h saja untuk "kira kira di sini"
	\centering 
	\caption{Tabel contoh}
	\label{tab:contoh2}
	\begin{tabular}{cccc}
		\toprule
		& $v_{start}$ & $\mathcal{S}_{1}$ & $v_{end}$\\

		\midrule
		$\tau_{1}$ & 1 & 12& 20\\
		$\tau_{2}$ & 1 &  & 20\\
		$\tau_{3}$ & 1 & 9 & 20\\
		$\tau_{4}$ & 1 &  & 20\\

		\bottomrule
		
	\end{tabular} 
\end{table}
Tabel~\ref{tab:cthwarna1} dan Tabel~\ref{tab:cthwarna2} berikut ini adalah tabel dengan sel yang berwarna dan ada dua tabel yang bersebelahan. 
\begin{table}[H]
	\begin{minipage}[c]{0.49\linewidth}
		\centering
		\caption{Tabel bewarna(1)}
		\label{tab:cthwarna1}
		\begin{tabular}{ccccc}
			\toprule
			 & $v_{start}$ & $\mathcal{S}_{2}$ & $\mathcal{S}_{1}$ & $v_{end}$\\
			
			\midrule
			$\tau_{1}$ & 1 & 5 \cellcolor{green}& 12& 20\\
			$\tau_{2}$ & 1 & 8 \cellcolor{green}& & 20\\
			$\tau_{3}$ & 1 & 2/8/17 \cellcolor{green}& 9 & 20\\
			$\tau_{4}$ & 1 & \cellcolor{red}& & 20\\
			
			\bottomrule

		\end{tabular}
	\end{minipage}
	\begin{minipage}[c]{0.49\linewidth}
		
		\centering 
		\caption{Tabel bewarna(2)}
		\label{tab:cthwarna2}
		\begin{tabular}{ccccc}
			\toprule
			 & $v_{start}$ & $\mathcal{S}_{1}$ & $\mathcal{S}_{2}$ & $v_{end}$\\
			
			\midrule
			$\tau_{1}$ & 1 & 12& 5 \cellcolor{red} &20\\
			$\tau_{2}$ & 1 &  &  8 \cellcolor{green} &20\\
			$\tau_{3}$ & 1 & 9 & 2/8/17 \cellcolor{green} &20\\
			$\tau_{4}$ & 1 &   & \cellcolor{red} &20\\
			
			\bottomrule
		
		\end{tabular}
	\end{minipage}
\end{table}

 
\subsection{Kutipan}
\label{subs:kutipan} 
Berikut contoh kutipan dari berbagai sumber, untuk keterangan lebih lengkap, silahkan membaca file referensi.bib yang disediakan juga di template ini.
Contoh kutipan:
\begin{itemize}
	\item Buku:~\cite{berg:08:compgeom} 
	\item Bab dalam buku:~\cite{kreveld:04:GIS}
	\item Artikel dari Jurnal:~\cite{buchin:13:median}
	\item Artikel dari prosiding seminar/konferensi:~\cite{kreveld:11:median}
	\item Skripsi/Thesis/Disertasi:~\cite{lionov:02:animasi}~\cite{wiratma:10:following}~\cite{wiratma:22:later}
	\item Technical/Scientific Report:~\cite{kreveld:07:watertight}
	\item RFC (Request For Comments):~\cite{RFC1654}
	\item Technical Documentation/Technical Manual:~\cite{Z.500}~\cite{unicode:16:stdv9}~\cite{google:16:and7}
	\item Paten:~\cite{webb:12:comm}
	\item Tidak dipublikasikan:~\cite{wiratma:09:median}~\cite{lionov:11:cpoly}
	\item Laman web:~\cite{erickson:03:cgmodel}  
	\item Lain-lain:~\cite{agung:12:tango}
\end{itemize}    
  
\subsection{Gambar}

Pada hampir semua editor, penempatan gambar di dalam dokumen \LaTeX{} tidak dapat dilakukan melalui proses {\it drag and drop}.
Perhatikan contoh pada file bab2.tex untuk melihat bagaimana cara menempatkan gambar.
Beberapa hal yang harus diperhatikan pada saat menempatkan gambar:
\begin{itemize}
	\item Setiap gambar {\bf harus} diacu di dalam teks (gunakan {\it field} {\sc label})
	\item {\it Field} {\sc caption} digunakan untuk teks pengantar pada gambar. Terdapat dua bagian yaitu yang ada di antara tanda $[$ dan $]$ dan yang ada di antara tanda $\{$ dan $\}$. Yang pertama akan muncul di Daftar Gambar, sedangkan yang kedua akan muncul di teks pengantar gambar. Untuk skripsi ini, samakan isi keduanya.
	\item Jenis file yang dapat digunakan sebagai gambar cukup banyak, tetapi yang paling populer adalah tipe {\sc png} (lihat Gambar~\ref{fig:ularpng}), tipe {\sc jpg} (Gambar~\ref{fig:ularjpg}) dan tipe {\sc pdf} (Gambar~\ref{fig:ularpdf})
	\item Besarnya gambar dapat diatur dengan {\it field} {\sc scale}.
	\item Penempatan gambar diatur menggunakan {\it placement specifier} (di antara tanda  $[$ dan $]$ setelah deklarasi gambar.
	Yang umum digunakan adalah {\bf H} untuk menempatkan gambar {\bf sesuai} penempatannya di file .tex atau  {\bf h} yang berarti "kira-kira" di sini. \\
	Jika tidak menggunakan {\it placement specifier}, \LaTeX{} akan menempatkan gambar secara otomatis untuk menghindari bagian kosong pada dokumen anda.
	Walaupun cara ini sangat mudah, hindarkan terjadinya penempatan dua gambar secara berurutan. 	
	\begin{itemize}
		\item Gambar~\ref{fig:ularpng} ditempatkan di bagian atas halaman, walaupun penempatannya dilakukan setelah penulisan 3 paragraf setelah penjelasan ini.
		\item Gambar~\ref{fig:ularjpg} dengan skala 0.5 ditempatkan di antara dua buah paragraf. Perhatikan penulisannya di dalam file bab2.tex!
		\item Gambar~\ref{fig:ularpdf} ditempatkan menggunakan {\it specifier} {\bf h}.
	\end{itemize}
\end{itemize}
 
\dtext{17-18}
\begin{figure} 
	\centering  
	\includegraphics[scale=1]{ular-png}  
	\caption[Gambar {\it Serpentes} dalam format png]{Gambar {\it Serpentes} dalam format png} 
	\label{fig:ularpng} 
\end{figure} 

\dtext{19-20}
\begin{figure}[H]
	\centering  
	\includegraphics[scale=0.5]{ular-jpg}  
	\caption[Ular kecil]{Ular kecil} 
	\label{fig:ularjpg} 
\end{figure} 
\dtext{21-22}

\begin{figure}[ht] 
	\centering  
	\includegraphics[scale=1]{ular-pdf}  
	\caption[ {\it Serpentes} betina]{ {\it Serpentes} jantan} 
	\label{fig:ularpdf} 
\end{figure} 
 
