\chapter{Kesimpulan dan Saran}
\label{chap:kesimpulan dan saran}

\section{Kesimpulan}
\label{sec:kesimpulan}

Berikut ini merupakan kesimpulan yang didapatkan berdasarkan penelitian yang sudah dilakukan :

\begin{enumerate}
    \item Sistem rekomendasi program studi Universitas Katolik Parahyangan sudah dapat dikembangankan dan memberikan rekomendasi kepada pengguna khususnya siswa/i SMA pada kelas 11 yang ingin melanjutkan pendidikan di Universitas Katolik Parahyangan.
    
    \item Sistem rekomendasi program studi Universitas Katolik Parahyangan yang dikembangkan dengan teknik \textit{user-based collaborative filtering} memberikan rekomendasi berdasaarkan \textit{rating} yang diberikan pengguna lain yang telah lulus dari Universitas Katolik Parahyangan.
    
    \item Hasil pengujian menggunakan metode \textit{Mean Absolute Error} (MAE), \textit{Root Mean Square Error} (RMSE), dan eksekusi waktu program pada metode dasar dan metode Kmeans dengan nilai MAE berada di 0.2, nilai RMSE berada di 0.3, dan waktu eksekusi program lebih lama pada metode KMeans, dikarenakan diperlukan membuat kelompok pada data \textit{train set}.
    
\end{enumerate}

\section{Saran}
\label{sec:saran}

Berdasarkan hasil penelitian yang dilakukan, penulis dapat memberikan beberapa saran sebagai berikut :

\begin{enumerate}
    \item Sistem rekomendasi program studi Universitas Katolik Parahyangan menggunakan data mahasiswa lulus pada jalur PMDK, untuk itu perlu ditambahkan data mahasiswa lulus pada jalur USM, dengan harapan semakin banyak jumlah pengguna yang memiliki kesanamaan dengan siswa/i SMA yang menggunakan sistem.
    
    \item Sistem rekomendasi program studi Universitas Katolik Parahyangan menggunakan data mahasiswa yang berasal dari Biro Teknologi Informasi (BTI) Universitas Katolik Parahyangan. Penulis berharap agar format penyimpanan nilai baik untuk jalur penerimaan PMDK dan UMS sama, dengan harapan mudah untuk digunakan pada sistem yang sudah dibangun.
    
    \item Teknik pengelompokkan yang digunakan adalah KMeans dimana saat pengujian membutuhkan waktu yang lebih lama, metode ini memberikan masalah pada waktu eksekusi program. Berdasarkan masalah ini, penulis berharap sistem dapat menggunakan metode lain untuk mengoptimalkan waktu eksekusi program.
\end{enumerate}