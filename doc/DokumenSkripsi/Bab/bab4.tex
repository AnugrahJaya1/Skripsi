\chapter{Perancangan}
\label{chap:perancangan}

\section{Perancangan Fisik Basis Data}

\subsection{Perancangan Tabel}
\label{sec:perancngan tabel}

Tabel data yang akan digunakan untuk sistem rekomendasi program studi Universitas Katolik Parahyangan sesuai diagram ERD pada gambar \ref{fig:diagram erd} akan dirancang sesuai pada tabel beriku :

\begin{enumerate}
    \item Jurusan SMA
    
        \begin{table}[H]
            \centering
            \begin{tabular}{|c|p{4cm}|p{4cm}|p{4cm}|}
                \hline
                No & Nama Atribut & Tipe Data & Keterangan \\
                \hline
                1 & id\_jurusan & INT & NOT NULL, Primary Key \\
                \hline
                2 & nama\_jurusan & VARCHAR(25) & NOT NULL\\
                \hline
            \end{tabular}
            \caption{Perancangan Tabel jurusan\_sma}
            \label{tab:perancangan tabel jurusan sma}
        \end{table}
    
    \item Fakultas
    
        \begin{table}[H]
            \centering
            \begin{tabular}{|c|p{4cm}|p{4cm}|p{4cm}|}
                \hline
                No & Nama Atribut & Tipe Data & Keterangan \\
                \hline
                1 & id\_fakultas & INT & NOT NULL, Primary Key \\
                \hline
                2 & nama\_fakultas & VARCHAR(50) & NOT NULL\\
                \hline
            \end{tabular}
            \caption{Perancangan Tabel fakultas}
            \label{tab:perancangan tabel fakultas}
        \end{table}
        
    \item Program Studi
        
        \begin{table}[H]
            \centering
            \begin{tabular}{|c|p{4cm}|p{4cm}|p{4cm}|}
                \hline
                No & Nama Atribut & Tipe Data & Keterangan \\
                \hline
                1 & id\_program\_studi & INT & NOT NULL, Primary Key \\
                \hline
                2 & nama\_program\_studi & VARCHAR(50) & NOT NULL \\
                \hline
                3 & id\_fakultas & INT & NOT NULL, Foreign Key dari fakultas \\
                \hline
                4 & id\_jurusan & INT & NOT NULL, Foreign Key dari jurusan\_sma \\
                \hline
            \end{tabular}
            \caption{Perancangan Tabel program\_studi}
            \label{tab:perancangan tabel program studi}
        \end{table}
        
    \item Mahasiswa
        
        \begin{table}[H]
            \centering
            \begin{tabular}{|c|p{4cm}|p{4cm}|p{4cm}|}
                \hline
                No & Nama Atribut & Tipe Data & Keterangan \\
                \hline
                1 & id\_mahasiswa & INT & NOT NULL, Primary Key \\
                \hline
                2 & NPM & VARCHAR(10) & NOT NULL \\
                \hline
                3 & IPK & DOUBLE & NOT NULL\\
                \hline
                4 & id\_jurusan & INT & NOT NULL, Foreign Key dari jurusan\\
                \hline
                5 & id\_program\_studi & INT & NOT NULL, Foreign Key  dari program\_studi \\
                \hline
            \end{tabular}
            \caption{Perancangan Tabel mahasiswa}
            \label{tab:perancangan tabel mahasiswa}
        \end{table}
        
    \item Mata Pelajaran
        
        \begin{table}[H]
            \centering
            \begin{tabular}{|c|p{4cm}|p{4cm}|p{4cm}|}
                \hline
                No & Nama Atribut & Tipe Data & Keterangan \\
                \hline
                1 & id\_mata\_pelajaran & INT & NOT NULL, Primary Key \\
                \hline
                2 & nama\_mata\_pelajaran & VARCHAR(20) & NOT NULL\\
                \hline
            \end{tabular}
            \caption{Perancangan Tabel mata\_pelajaran}
            \label{tab:perancangan tabel mata pelajaran}
        \end{table}
        
    \item Nilai
    
        \begin{table}[H]
            \centering
            \begin{tabular}{|c|p{4cm}|p{4cm}|p{4cm}|}
                \hline
                No & Nama Atribut & Tipe Data & Keterangan \\
                \hline
                1 & id\_nilai & INT & NOT NULL, Primary Key \\
                \hline
                2 & id\_mata\_pelajaran & INT & Foreign Key dari mata\_pelajaran \\
                \hline
                3 & id\_mahasiswa & INT & NOT NULL, Foreign Key dari mahasiswa \\
                \hline
                4 & 101 & DOUBLE & Nilai kelas 10 semester 1 \\
                \hline
                5 & 102 & DOUBLE & Nilai kelas 10 semester 2 \\
                \hline
                6 & 111 & DOUBLE & Nilai kelas 11 semester 1 \\
                \hline
                7 & 112 & DOUBLE & Nilai kelas 11 semester 2 \\
                \hline
                8 & AVG & DOUBLE & Rata-rata nilai \\
                \hline
            \end{tabular}
            \caption{Perancangan Tabel nilai}
            \label{tab:perancangan tabel nilai}
        \end{table}
\end{enumerate}

\section{Perancangan Algoritma}
\label{sec:perancangan algoritma}

Pada subab ini akan berisikan perancangan algoritma yang digukan pada sistem. Sistem menggunakan beberapa algoritma seperti K-Means untuk membuat kelompok mahasiswa yang memiliki karakteriksi yang sama dengan calon mahasiswa, Pearson Correlation Coefficient untuk menghitung kesamaan atau similaritas, dan User-base Collaborative Filtering untuk menghitung prediksi nilai IPK.

\subsection{Mahasiswa Controller}
\label{subsec:mahasiswa controller}

\begin{algorithm}[H]
  \begin{algorithmic}[1]
    \Procedure{indek}{$jurusanSMA$} 
        \State $idJurusan \gets 1$ \Comment{IdJurusan IPA}
        \If{jurusanSMA == 'IPS'}
            \State $idJurusan \gets 2$
        \EndIf
        
        \State $dataMahasiswa \gets DATAMAHASISWA(idJurusan)$
        
        \State \Return dataMahasiswa
    \EndProcedure
  \end{algorithmic} 
  \caption{Index}
  \label{alg:index mahasiswa controller}
\end{algorithm}

\begin{algorithm}[H]
  \begin{algorithmic}[1]
    \Procedure{dataMahasiswa}{$idJurusan$} 
        \State $query \gets SELECT * FROM mahasiswa INNER JOIN nilai ON mahasiswa.id\_mahasiswa=nilai.id\_mahasiswa WHERE id\_jurusan = idJurusan$
        
        \State \Return query
    \EndProcedure
  \end{algorithmic} 
  \caption{Data Mahasiswa}
  \label{alg:data mahasiswa controller}
\end{algorithm}

\subsection{Siswa Controller}
\label{subsec:siswa controller}

\begin{algorithm}[H]
  \begin{algorithmic}[1]
    \Procedure{index}{$request$} 
        \State $data \gets request.INPUT()$
        \State $DATASISWA(data)$
        
        \State $mahasiswa \gets MAHASISWACONTROLLER()$
        \State $mhs \gets mahasiswa.INDEX(siswa["btn"]).TOARRAY()$
        
        \State $kmeans \gets KMEANSCONTROLLER(k, mhs)$ \Comment{k adalah jumlah kelompok yang ingin dibentuk}
        \State $cluster \gets kmeans.HITUNGJARAKSISWA(siswa)$
        
        \State $mhs \gets kmeans.GETCLUSTER(cluster)$
        
        \State $userBasedModel \gets USERBASEDMODELCONTROLLER(mhs, siswa)$
        
        \State $result \gets userBasedModel.GETRESULT()$
        
        \State \Return $view('/result', ['result' \gets result])$
    \EndProcedure
  \end{algorithmic} 
  \caption{Index}
  \label{alg:index siswa controller}
\end{algorithm}

\begin{algorithm}[H]
  \begin{algorithmic}[1]
    \Procedure{dataSiswa}{$data$} 
        \State $i \gets 1$
        \State $result \gets ARRAY()$
        \State $result['nilai'] \gets ARRAY()$
        
        \ForEach{$key => value \in data$}
            \If{key == '\_token'}
                \State $result[key] \gets value$
            \Else
                \If{i == 1}
                    \State $k \gets SUBSTR(key, 0, 3)$
                    \State $temp \gets ARRAY()$
                    
                    \State ARRAY\_PUSH(temp, ((int)value/20)-1) \Comment{Conver kedalam GPA}
                    
                    \State $i \gets i+1$
                \Else
                    \State ARRAY\_PUSH(temp, ((int)value/20)-1) \Comment{Conver kedalam GPA}
                    
                    \State $i \gets i+1$
                    
                    \If{i == 5}
                        \State $avg \gets ARRAY\_SUM(temp)/COUNT(temp)$
                        \State ARRAY\_PUSH(temp, avg)
                        
                        \State $temp \gets REPLACEKEY(temp, 5, 'id\_mata\_pelajaran')$
                        
                        \State ARRAY\_PUSH(result['nilai',temp)
                        
                        \State $i \gets 1$
                    \EndIf
                \EndIf
            \EndIf
        \EndFor
        
        \If{!EMPTY(data['btnIPA'])}
            \State $result['btn'] \gets 'IPA'$
        \ElsIf{!EMPTY(data['btnIPS'])}
            \State $result['btn'] \gets 'IPS'$
        \EndIf
        
        \State \Return result
    \EndProcedure
  \end{algorithmic} 
  \caption{Data Siswa}
  \label{alg:data siswa controller}
\end{algorithm}

\begin{algorithm}[H]
  \begin{algorithmic}[1]
    \Procedure{ReplaceKey}{$temp, oldKey, newKey$} 
        \State $temp[newKey] \gets temp[oldKey]$
        \State UNSET(temp[oldKey])
        
        \State \Return temp
    \EndProcedure
  \end{algorithmic} 
  \caption{Replace Key}
  \label{alg:Replace Key siswa controller}
\end{algorithm}

\subsection{K-Means Controller}
\label{subsec:kmeans}

\begin{algorithm}[H]
  \begin{algorithmic}[1]
    \Procedure{Contructor}{$k, dataMahasiswa$} 
        \State $k \gets k$
        \State $mahasiswa \gets dataMahasiswa$
        \State $INISIALISASICLUSTER()$
        \State $currCentroid \gets ARRAY()$ \Comment{centroid saat ini}
        
        \State $J0 \gets 100$ \Comment{J0 = inisialisasi jarak total dari objek ke centroid-nya}
        
        \State $PILIHCENTROID()$
        
        \State $HITUNGJARAKMHS()$
        
        \State $status \gets TRUE$
        \State $ idx \gets 0$
        \While{status}
            \State HITUNGCENTROIDBARU()
            \State $idx \gets idx + 1$
            \State $status \gets CEKBATAS()$
            \State $HITUNGJARAKMHS()$
        \EndWhile
    \EndProcedure
  \end{algorithmic} 
  \caption{Contructor KMeans}
  \label{alg:contructor kmeans}
\end{algorithm}

\begin{algorithm}[H]
  \begin{algorithmic}[1]
    \Procedure{inisialisasiCluster}{} 
        \State $cluster \gets ARRAY()$
        \For{i = 1 into k}
            \State $cluster[i] \gets ARRAY()$
        \EndFor
    \EndProcedure
  \end{algorithmic} 
  \caption{Inisialisasi Cluster}
  \label{alg:inisialisasi cluster}
\end{algorithm}

\begin{algorithm}[H]
  \begin{algorithmic}[1]
    \Procedure{pilihCentroid}{} 
        \State $i \gets 0$
        \While{i < k}
            \State $key \gets RAND(0,1739)$ \Comment{Random sebanyak jumlah mahasiswa}
            
            \If{check key in mahasiswa == TRUE}
                \If{check key in mahasiswa == FALSE}
                    \State ARRAY\_PUSH(currCentroid, mahasiswa[key])
                    \State $i \gets i+1$
                \EndIf
            \EndIf
        \EndWhile
    \EndProcedure
  \end{algorithmic} 
  \caption{Pilih Centroid}
  \label{alg:pilih centroid}
\end{algorithm}


\begin{algorithm}[H]
  \begin{algorithmic}[1]
    \Procedure{hitungJarakMhs}{} 
        \State $J1 \gets 0$
        \ForEach{$valueMhs \in mahasiswa$}
            \State $temoCluster \gets ARRAY()$ \Comment{Penampung cluster sementara}
            \State $nilaiMhs \gets valueMhs['nilai']$
            
            \ForEach{$valueNilaiMhs \in nilaiMhs$}
                \State $arrayJarak \gets ARRAY()$
                
                \ForEach{$valueNilaiCen \in currCentroid$}
                    \If{$(valueNilaiMhs['id_mata_pelajaran'] == 1 AND valueNilaiCen['id_mata_pelajaran'] == 1) OR (valueNilaiMhs['id_mata_pelajaran'] == 3 AND valueNilaiCen['id_mata_pelajaran'] == 3)$}
                        \State $jarak \gets EUCLIDIANCEDISTANCE(valueNilaiMhs, valueNilaiCen)$
                    \ElsIf{$valueNilaiMhs['id_mata_pelajaran'] < valueNilaiCen['id_mata_pelajaran']$}
                        \State break
                    \EndIf
                \EndFor
                \State ARRAY\_PUSH(arrJarak, jarak)
            \EndFor
            \If{$tempCluster is empty$}
                \State ARRAY\_PUSH(tempCluster, arrJarak)
            \Else
                \For{i = 1 into k}
                    \State $tempCluster[0][i] \gets tempCluster[0][i]+ arrJarak[i]$
                    \State $tempCluster[0][i] \gets SQRT(tempCluster[0][i])$
                \EndFor
            \EndIf
            \State $c \gets current mahasiswa cluster$
            \State $J1 \gets J1 + tempCluster[0][c]$
            \State ARRAY\_PUSH(tempCluster[0], c, valueMhs['id\_mahasiswa'])
            
            \State $tempCluster[0]['id\_mahasiswa'] \gets tempCluster[0][k+1]$
            \State UNSET(tempCluster[0][k+1])
            
            \State ARRAY\_PUSH(cluster[c],valueMhs)
        \EndFor
    \EndProcedure
  \end{algorithmic} 
  \caption{Hitung Jarak Mhs}
  \label{alg:hitungJarakMhs}
\end{algorithm}

\begin{algorithm}[H]
  \begin{algorithmic}[1]
    \Procedure{hitungJarakSiswa}{siswa} 
        \State $nilaiSiswa \gets siswa['nilai']$
        \State $tempCluster \gets ARRAY()$
        
        \ForEach{$valueNilaiSiswa \in nilaiSiswa$}
            \State $arrJarak \gets ARRAY()$
            
            \ForEach{$valueCen \in currCentroid$}
                \State $jarak \gets 0$
                \State $nilaiCen \gets valueCen['nilai']$
                
                \ForEach{$valueNilaiCen \in nilaiCen$}
                    \If{$(valueNilaiMhs['id_mata_pelajaran'] == 1 AND valueNilaiCen['id_mata_pelajaran'] == 1) OR (valueNilaiMhs['id_mata_pelajaran'] == 3 AND valueNilaiCen['id_mata_pelajaran'] == 3)$}
                    \State $jarak \gets EUCLIDIANCEDISTANCE(valueNilaiSiswa, valueNilaiCen)$
                    \ElsIf{$valueNilaiMhs['id_mata_pelajaran'] < valueNilaiCen['id_mata_pelajaran']$}
                        \State break
                    \EndIf
                \EndFor
                \State ARRAY\_PUSH(arrJarak,jarak)
            \EndFor
            \If{$tempCluster is empty$}
                \State ARRAY\_PUSH(tempCluster, arrJarak)
            \Else
                \For{i = 1 into k}
                    \State $tempCluster[0][i] \gets tempCluster[0][i]+ arrJarak[i]$
                    \State $tempCluster[0][i] \gets SQRT(tempCluster[0][i])$
                \EndFor
            \EndIf
        \EndFor
        
        \State $res \gets current siswa cluster$
        
        \State \Return res
    \EndProcedure
  \end{algorithmic} 
  \caption{Hitung Jarak Siswa}
  \label{alg:hitungJarakSiswa}
\end{algorithm}

\begin{algorithm}[H]
  \begin{algorithmic}[1]
    \Procedure{euclidianceDistance}{mhs, centroid} 
        \State $result \gets 0$
        
        \For{i = 1 into 4}
            \State $result \gets result + POW(mhs[i]-centroid[i],2)$
        \EndFor
        \State $result \gets result + POW(mhs['AVG']-centroid['AVG'],2)$
        
        \State \Return result
    \EndProcedure
  \end{algorithmic} 
  \caption{Euclidiance Distance}
  \label{alg:euclidianceDistance}
\end{algorithm}

\begin{algorithm}[H]
  \begin{algorithmic}[1]
    \Procedure{hitungCentroidBaru}{} 
        \State $prevCentroid \gets currCentroid$
        \State RESETCENTROID()
        
        \ForEach{$keyCen=>valueCen \in currCentroid$}
            \State $nilaiCen \gets valueCen['nilai']$
            
            \ForEach{$keyNilaiCen=>valueNilaiCen \in nilaiCen$}
                \State $anggota \gets cluster[keyCen]$
                \If{numbers of anggota != 0}
                    \ForEach{$keyAnggota=>valueAnggota \in anggota$}
                        \State $nilaiAnggota = valueAnggota['nilai']$
                        
                        \ForEach{$keyNilaiAnggota=>valueNilaiAnggota \in nilaiAnggota$}
                            \If{$(valueNilaiMhs['id_mata_pelajaran'] == 1 AND valueNilaiCen['id_mata_pelajaran'] == 1) OR (valueNilaiMhs['id_mata_pelajaran'] == 3 AND valueNilaiCen['id_mata_pelajaran'] == 3)$}
                                \For{i = 1 into 4}
                                    \State $nilaiLama \gets currCentroid[keyCen]['nilai][keyNilaiCen][i]$
                                    \State $nilaiBaru \gets anggora[keyAnggota]['nilai'][keyNilaiAnggota][i]$
                                    
                                    \State UPDATENILAI(keyCen, keyNilaiCen, nilaiLama, nilaiBaru, i)
                                \EndFor
                                \State $nilaiLama \gets currCentroid[keyCen]['nilai][keyNilaiCen]['AVG']$
                                \State $nilaiBaru \gets anggora[keyAnggota]['nilai'][keyNilaiAnggota]['AVG']$
                                \State UPDATENILAI(keyCen, keyNilaiCen, nilaiLama, nilaiBaru, 'AVG')
                            \EndIf
                        \EndFor
                    \EndFor
                    \Else
                        \State RANDOMNILAIBARU(keyCen, KeyNilaiCen)
                \EndIf
            \EndFor
        \EndFor
        \State HITUNGRATA2()
    \EndProcedure
  \end{algorithmic} 
  \caption{Hitung Centroid Baru}
  \label{alg:hitungCentroidBaru}
\end{algorithm}

\begin{algorithm}[H]
  \begin{algorithmic}[1]
    \Procedure{resetCentroid}{} 
        \ForEach{$keyCen => valueCen \in currCentroid$}
            \State $nilaiCen \gets valueCen['nilai']$
            
            \ForEach{$keyNilai \in nilaiCen$}
                \For{i = 1 into 4}
                    \State $currCentroid[keyCen]['nilai'][keyNilaiCen][i] \gets 0$
                \EndFor
                \State $currCentroid[keyCen]['nilai'][keyNilaiCen]['AVG'] \gets 0$
            \EndFor
        \EndFor
    \EndProcedure
  \end{algorithmic} 
  \caption{Reset Centroid}
  \label{alg:resetCentroid}
\end{algorithm}

\begin{algorithm}[H]
  \begin{algorithmic}[1]
    \Procedure{updateNilai}{keyCen, keyNilaiCen, nilaiLama, nilaiBaru, i} 
      \State $nilai \gets nilaiLama+nilaiBaru$
      \State $currCentroid[keyCen]['nilai'][keyNilaiCen][i] \gets nilai$
    \EndProcedure
  \end{algorithmic} 
  \caption{Update Nilai}
  \label{alg:updateNilai}
\end{algorithm}

\begin{algorithm}[H]
  \begin{algorithmic}[1]
    \Procedure{hitungRata2}{} 
        \ForEach{$keyCen => valueCen \in currCentroid$}
            \State $nilaiCen \gets valueCen['nilai']$
            \State $anggota \gets cluster[keyCen]$
            \State $count \gets numbers of anggota$
            \If{counter != 0}
                \ForEach{$keyNilaiCen \in nilaiCen$}
                    \For{i = 1 into 4}
                        \State $currCentroid[keyCen]['nilai'][keyNilaiCen][i] \gets currCentroid[keyCen]['nilai'][keyNilaiCen][i]/count$
                    \EndFor
                    $currCentroid[keyCen]['nilai'][keyNilaiCen]['AVG'] \gets currCentroid[keyCen]['nilai'][keyNilaiCen]['AVG']/count$
                \EndFor
            \EndIf
        \EndFor
    \EndProcedure
  \end{algorithmic} 
  \caption{Hitung Rata2}
  \label{alg:hitungRata2}
\end{algorithm}

\begin{algorithm}[H]
  \begin{algorithmic}[1]
    \Procedure{randomNilaiBaru}{keyCen, keyNilaiCen} 
        \For{i = 1 into 4}
            \State $currCentroid[keyCen]['nilai'][keyNilaiCen][i] \gets RAND(1,3) + RAND(1,10)/10$
        \EndFor
         \State $currCentroid[keyCen]['nilai'][keyNilaiCen]['AVG'] \gets RAND(1,3) + RAND(1,10)/10$
    \EndProcedure
  \end{algorithmic} 
  \caption{Random NilaiBaru}
  \label{alg:randomNilaiBaru}
\end{algorithm}

\begin{algorithm}[H]
  \begin{algorithmic}[1]
    \Procedure{cekBatas}{} 
        \State $batas \gets ABS(J0 - J1)$
        
        \If{batas < 0.1}
            \State \Return FALSE
        \EndIf
        
        \State \Return TRUE
    \EndProcedure
  \end{algorithmic} 
  \caption{Cek Batas}
  \label{alg:cekBatas}
\end{algorithm}

\begin{algorithm}[H]
  \begin{algorithmic}[1]
    \Procedure{getCluster}{idx} 
        \State \Return cluster[idx]
    \EndProcedure
  \end{algorithmic} 
  \caption{Get Cluster}
  \label{alg:getCluster}
\end{algorithm}

\subsection{User-based Collaborative Filtering Controller}
\label{subsec:user-based}

\begin{algorithm}[H]
  \begin{algorithmic}[1]
    \Procedure{Contructor}{$mahasiswa, siswa, mode = 0$} 
        \State $prediction \gets PREDICTIONCONTROLLER()$
        \If {mode == 0}
            \State $pearsonCorrelation \gets PEARSONCORRELATIONCONTROLLER()$
            \State $pearson \gets CALCULATESIMILARITY (mahasiswa, siswa)$
            \State $result \gets CALCULATEPREDICT(pearson)$
        \ElsIf {mode == 1}
            \State $pearsonCorrelation \gets PEARSONCORRELATIONPENGUJIANCONTROLLER()$
        \EndIf
    \EndProcedure
  \end{algorithmic} 
  \caption{User-based Collaborative Filtering}
  \label{alg:contructor user-based}
\end{algorithm}

\begin{algorithm}[H]
  \begin{algorithmic}[1]
    \Procedure{calculateSimilarity}{$mahasiswa, siswa$} 
        \State \Return $pearsonCorrelation \gets CALCULATEPEARSON(mahaiswa, siswa)$
    \EndProcedure
  \end{algorithmic} 
  \caption{User-based Collaborative Filtering}
  \label{alg:calculateSimilarity user-based}
\end{algorithm}

\begin{algorithm}[H]
  \begin{algorithmic}[1]
    \Procedure{calculatePredict}{$pearson$} 
        \State \Return $prediction \gets CALCULATEPREDICT(pearson)$
    \EndProcedure
  \end{algorithmic} 
  \caption{User-based Collaborative Filtering}
  \label{alg:calculatePredict user-based}
\end{algorithm}

\begin{algorithm}[H]
  \begin{algorithmic}[1]
    \Procedure{getResult}{} 
        \State \Return result
    \EndProcedure
  \end{algorithmic} 
  \caption{User-based Collaborative Filtering}
  \label{alg:getResult user-based}
\end{algorithm}

\subsection{Pearson Correlation Coefficient Controller}
\label{subsec:pearson}

\begin{algorithm}[H]
  \begin{algorithmic}[1]
    \Procedure{calculatePearson}{$mahasiswa, siswa$} 
        \State $res \gets ARRAY()$
        \ForEach{$mahasiswa \in  mhs $}
            \State $covariance \gets CALCULATECOVARIANCE(mhs, siswa)$
            \State $sd \gets CALCULATESTANDARDEVIATION(mhs,siswa)$
            \State $sdMhs \gets sd[0]$
            \State $sdSiswa \gets sd[1]$
            
            \State $idProdi \gets mhs['id_program\_studi']$
            \State $IPK \gets mhs['IPK']$
            
            \State $sim \gets CONVARIANCE / (sdMhs * sdSiswa)$
            
            \If{$sim > 0$}
                \State $res[mhs['id\_mahasiswa']] \gets ARRAY()$
                \State $ARRAY\_PUSH(res[mhs['id\_mahasiswa']], sim, idProdi, IPK)$
            \EndIf
        \EndFor
    
        \State \Return res
    \EndProcedure
  \end{algorithmic} 
  \caption{Pearson Correlation Coefficient}
  \label{alg:calculatePearson pearson}
\end{algorithm}

\begin{algorithm}[H]
  \begin{algorithmic}[1]
    \Procedure{calculateCovariance}{$mhs, siswa$} 
        \State $res \gets 0$
        \State $nilaiMhs \gets mhs['nilai']$
        \State $nilaiSiswa \gets siswa['nilai']$
        
        \ForEach{$nilaiSiswa \in  nSiswa $}
            \State $idMP \gets nSiswa['id\_mata\_pelajaran']$
            \ForEach{$nilaiMhs \in  nMhs $}
                \If{idMP == nMhs['id\_mata\_pelajaran']}
                    \For{i = 1 to 4}
                        \State $res \gets res +  (nMhs[i]-nMhs['AVG'])*(nSiswa[i]-nSiswa['AVG'])$
                    \EndFor
                \ElsIf{idMP < nMhs['id\_mata\_pelajaran']}
                    \State break
                \EndIf
            \EndFor
        \EndFor
        \State \Return res
    \EndProcedure
  \end{algorithmic} 
  \caption{Pearson Correlation Coefficient}
  \label{alg:calculateCovariance pearson}
\end{algorithm}

\begin{algorithm}[H]
  \begin{algorithmic}[1]
    \Procedure{calculateStandarDeviation}{$mhs, siswa$} 
        \State $res \gets 0$
        
        \State $sdMhs \gets 0$
        \State $sdSiswa \gets 0$
        
        \State $nilaiMhs \gets mhs['nilai']$
        \State $nilaiSiswa \gets siswa['nilai']$
        
        \ForEach{$nilaiSiswa \in nSiswa$}
            \State $idMP \gets nSiswa['id\_mata\_pelajaran']$
            \ForEach{$nilaiMhs \in  nMhs $}
                \If{idMP == nMhs['id\_mata\_pelajaran']}
                    \For{i = 1 to 4}
                        \State $sdMhs \gets sdMhs + POW(nMhs[i]-nMhs['AVG'],2)$
                        \State $sdSiswa \gets sdSiswa + POW(nSiswa[i]-nSiswa['AVG'],2)$
                    \EndFor
                \ElsIf{idMP < nMhs['id\_mata\_pelajaran']}
                    \State break
                \EndIf
            \EndFor
        \EndFor
        \State $ARRAY\_PUSH(res, SQRT(sdMhs), SQRT(sdSiswa))$
        
        \State \Return res
    \EndProcedure
  \end{algorithmic} 
  \caption{Pearson Correlation Coefficient}
  \label{alg:calculateStandarDeviation pearson}
\end{algorithm}


\subsubsection{Precidtion Controller}
\label{subsec:prediction}

\begin{algorithm}[H]
  \begin{algorithmic}[1]
    \Procedure{Contructor}{} 
        \State $programStudi \gets PROGRAMSTUDICONTROLLER()$
        \State $fakultas \gets FAKULTASCONTROLLER()$
    \EndProcedure
  \end{algorithmic} 
  \caption{Prediction}
  \label{alg:contructor prediction}
\end{algorithm}

\begin{algorithm}[H]
  \begin{algorithmic}[1]
    \Procedure{calculatePredict}{$pearson$} 
        \State $res \gets ARRAY()$
        
        \State $a \gets 0$ \Comment{a = Sigma(sim*IPK)}
        \State $b \gets 0$ \Comment{b = Sigma(sim)}
        
        \ForEach{$pearson \in  id\_mhs => value $}
            \State $a \gets a + value[0]*value[2]$
            \State $b \gets b + value[0]$
            
            \State $next \gets NEXT(pearson)$
            
            \If{$next != NULL$}
                \If{$value[1] != next[1]$}  
                    \State $res \gets INSERTDATA(res, a, b, value[1])$
                    
                    \State $a \gets 0$
                    \State $b \gets 0$
                \EndIf
            \ElsIf{$next != NULL$}
                \State $res \gets INSERTDATA(res, a, b, value[1])$
            \EndIf
        \EndFor
        
        \State $score \gets ARRAY_COLUMN(res,0)$ \Comment{Penampung nilai prediksi IPK}
        \State $ARRAY\_MULTISORT(score, SORT\_DESC, res)$ \Comment{Sort berdasarkan nilai prediksi terbesar}
        
        \State \Return res
    \EndProcedure
  \end{algorithmic} 
  \caption{Prediction}
  \label{alg:calculatePredict prediction}
\end{algorithm}

\begin{algorithm}[H]
  \begin{algorithmic}[1]
    \Procedure{insertData}{$res, a, b, idProdi$} 
        \State $pred \gets a/b$
        \State $namaFakultas \gets fakultas.GETNAMAFAKULTAS(idProdi)$
        \State $namaProdi \gets programStudi.GETNAMAPROGRAMSTUDI(idProdi)$
        \State $res[idProdi] \gets ARRAY()$
        \State $ARRAY\_PUSH(res[idProdi], pred, namaFakultas, namaProdi)$
        \State \Return res
    \EndProcedure
  \end{algorithmic} 
  \caption{Prediction}
  \label{alg:insertData prediction}
\end{algorithm}


\section{Perancangan Antar Muka}
\label{sec:perancangan antar muka}

Pada subab ini akan berisikan perancangan antar muka untuk sistem rekomendasi, berikut merupakan hasil perancangan :

\begin{enumerate}
    \item Halaman index saat siswa/i mengakses sistem
    
    \begin{figure}[H]
        \centering
        \includegraphics[width = 12cm, height =8 cm]{doc/DokumenSkripsi/Gambar/gambar41.png}
        \caption{Halaman Index Sistem}
        \label{fig:gambar41}
    \end{figure}
    
    \item Halaman pengisian nilai siswa/i IPA
    
    \begin{figure}[H]
        \centering
        \includegraphics[width = 12cm, height =8 cm]{doc/DokumenSkripsi/Gambar/gambar42.png}
        \caption{Halaman Pengisian Nilai IPA}
        \label{fig:gambar42}
    \end{figure}
    
    \item Halaman pengisian nilai siswa/i IPS
    
    \begin{figure}[H]
        \centering
        \includegraphics[width = 12cm, height =8 cm]{doc/DokumenSkripsi/Gambar/gambar43.png}
        \caption{Halaman Pengisian Nilai IPS}
        \label{fig:gambar43}
    \end{figure}
    
    \item Halaman hasil rekomendasi
    
    \begin{figure}[H]
        \centering
        \includegraphics[width = 12cm, height =8 cm]{doc/DokumenSkripsi/Gambar/gambar44.png}
        \caption{Halaman Hasil Rekomendasi}
        \label{fig:gambar44}
    \end{figure}
\end{enumerate}