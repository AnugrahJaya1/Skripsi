%versi 3 (18-12-2016)
\chapter{Kode Program}
\label{lamp:A}

%terdapat 2 cara untuk memasukkan kode program
% 1. menggunakan perintah \lstinputlisting (kode program ditempatkan di folder yang sama dengan file ini)
% 2. menggunakan environment lstlisting (kode program dituliskan di dalam file ini)
% Perhatikan contoh yang diberikan!!
%
% untuk keduanya, ada parameter yang harus diisi:
% - language: bahasa dari kode program (pilihan: Java, C, C++, PHP, Matlab, C#, HTML, R, Python, SQL, dll)
% - caption: nama file dari kode program yang akan ditampilkan di dokumen akhir
%
% Perhatian: Abaikan warning tentang textasteriskcentered!!
%

% kode untuk mahasiswa
\lstinputlisting[language=PHP, caption=MahasiswaController.php]{./Lampiran/MahasiswaController.php}

% Kode untuk siswa controller
\lstinputlisting[language=PHP, caption=SiswaController.php]{./Lampiran/SiswaController.php}

% kode untuk kmeans controller
\lstinputlisting[language=PHP, caption=KMeansController.php]{./Lampiran/KMeansController.php}

% kode untuk user-based
\lstinputlisting[language=PHP, caption=UserBasedModelController.php]{./Lampiran/UserBasedModelController.php} 

% kode untuk pearson
\lstinputlisting[language=PHP, caption=PearsonCorrelationController.php]{./Lampiran/PearsonCorrelationController.php} 

% kode untuk prediksi
\lstinputlisting[language=PHP, caption=PredictionController.php]{./Lampiran/PredictionController.php} 

